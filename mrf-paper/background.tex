\section{Background}\label{sec:background}

First we review Amdahl's scaling law and then we show a typical pattern to surpass it: distributed caching.

\subsection{Amdahl's Law}
\label{sec:amdahl-law}

A cornerstone result in parallel computing, Amdahl's law \cite{10.1145/1465482.1465560} establishes a firm limit on the performance gain one can obtain by distributing a computation task over multiple processors. Given a partially parallel program, denote the fraction of execution time spent % by a single-threaded execution
% by the processor 
in the serial part of the code by $s$, and the parallel fraction by $(1-s)$. Here, some code is ``serial'' if it cannot benefit from the improvement of the parallel computing resources, like single-threaded code, critical sections guarded by exclusion locks, etc. Denote by $T(k)$ the runtime (in seconds) of the program when executed on $k$ processors, and let $S(k)=\frac{T(1)}{T(k)}$ denote the performance improvement relative to a single-threaded execution (i.e., the \emph{speedup}). Then, the following relation holds (see Fig.~\ref{fig:amdahl}):
\begin{equation}\label{eq:amdahl}
S(k) = \frac{T(1)}{T(k)} = \frac{1}{s + \frac{1-s}{k}} \enspace .
\end{equation}

Here, the term $\frac{1-s}{k}$ establishes that the perfectly parallel part of the program executes $k$ times faster on $k$ processors than on a single core. By Amdahl's law, \emph{(i)} no code can scale faster than linear (i.e., $\frac{d S(k)}{d k} \le 1$, with equality exactly when $s=0$), \emph{(ii)} throwing additional workers on a computation task yields diminishing returns ($\frac{d S(k)}{d k}$ is decreasing in $k$) and \emph{(iii)} the asymptotics is limited by the serial part only ($\lim_{k\to \infty}S(k) = \frac1{s}$). 

% \subsection{Superlinear speedup}
% \label{sec:backgound-superlinear}

% The uniting idea of both parallel computing and multi-robot systems is that having multi-
% ple processors or robots working on a task decreases the processing time. Typically we desire
% a linear speedup, that is, doubling the number of processing units halves the execution time.
% Sometimes superlinear scalability is observed in parallel computing systems and more fre-
% quently in multi-robot and swarm systems. Superlinearity means each individual processing
% unit gets more efficient by increasing the system size—a desired and rather counterintuitive
% phenomenon.

% As superlinear speedups seem special, they were frequently discussed and studied [2, 3]. There
% even exists a proof showing the impossibility of superlinear speedups but it assumes fixed problem
% size [4]. Superlinear speedups are rather infrequently observed in parallel computing (e.g., cache-
% size effects [5]) compared to rather frequent observations in multi-robot and swarm systems (e.g.,
% inherently collaborative tasks [6]). When observed, superlinearity is often a discrete effect, such
% as a workpackage happening to fit into the processors cache [5] or a robot group being able to
% form a bucket brigade [7, 8]. Superlinear scalability has much potential that should be enough
% motivation to investigate it across different domains and to understand how one can provoke it.

% Many authors reported the existence of a superlinear
% speedup, but most of them only mentioned it as a side effect
% [4]. Besides reporting a superlinearity, other researchers briefly
% presented that the reason for achieving a superlinear speedup
% is because of the greater amount of cache memory in the
% parallel execution compared to the sequential [5].

\subsection{Distributed Caching}
\label{sec:backgound-dist-cache}

Critical to Amdahl's law is the assumption that the size of the individual sub-problems assigned to the workers remains constant as we scale the system (see also ``Gustafson's law'' \cite{10.1145/42411.42415}). Under this ``fixed size'' assumption \cite{556383}, faster-than-linear scaling is impossible \cite{10.1016/0167-8191(86)90024-4}. However, when this assumption fails, say, when the problem size gets smaller or execution gets progressively faster as we add more parallel workers (``scaled size'' model \cite{556383}), superlinear\footnote{In this context, any function growing faster than $f(x) = x$ is considered ``superlinear'', despite that, for instance, $f(x) = 3x$ is still, mathematically speaking, linear. Some authors distinguish these functions using the term ``superunitary'' \cite{80148}. In line with the literature we will use the former terminology in the sequel.} scaling is often observed \cite{scalability-analyzed, sdn-analytitcs, 6483679, 10.1007/978-3-319-77610-1, dobb-1, dobb-2}. The most prominent example for the scaled size model is \emph{distributed caching} \cite{scalability-analyzed, sdn-analytitcs, dobb-2} (for complete taxonomies see \cite{556383, 7733347, 80148}).

% there is an actual conclusion in `` SUPERLINEAR SPEEDUP IN HPC SYSTEMS: WHY AND WHEN?``: ``Mainly the superlinear speedup performance in persistent algorithms occurs due to the increased cache re- sources in the parallel computer architectures, the prefetching of shared variables in shared memory organization, or better scheduling in heterogeneous environments.''

% is can happen, for instance, due intricate interplays between per-worker problem size and available memory in certain specific applications, like distributed matrix multiplication and factorization \cite{6483679, 80148, 7733347}. In other cases superlinearity emerges out of pure luck, say because the code happens to progressively spend more time in faster subroutines as it is being scaled \cite{556383} or a parallel search algorithm finishes faster on some specific input \cite{80148}. And there are cases when superlinearity is observed due to simple benchmarking errors \cite{gunther-hotsos,10.1145/2773212.2789974}. But genuine superlinearity is most often observed in the context of 

Most modern CPUs come with unshared Level-1 fast cache memory: the more CPU cores the more the cache memory and the smaller the per-worker subproblems. This improves cache-hit rate, which tends to speed up memory\slash cache-bound code disproportionately. Many distributed applications explicitly contain a fast-path\slash cache; e.g., \texttt{memcached} is often used as a fast cache for a ``slow'' web service \cite{180324,10.5555/1012889.1012894}, popular keys are cached in the OS kernel for fast key-value store access \cite{179747, ghigoff2021bmc}, FIB caches maintain the most recent IP routes to sidestep longest prefix matching \cite{rottenstreich2016optimal}, hierarchical flow caches serve as a fast-path in programmable software switches \cite{188960}, etc. All these workloads may benefit from the caches becoming more efficient as the system is scaled. % We stress, however, that faster-than-linear speedup is strictly contingent on the way work is distributed across workers so that subproblem sizes indeed reduce, otherwise cache efficiency remains constant and superlinear growth vanishes (see later).

\begin{figure}
  \centering
  \begin{small}
    \begin{small}
  \tikzmath
  {
    function lookup(\x)
    {
      if (\x < 10) then
      {
        return 0.1+0.9*(0.1*\x +(1-0.1*\x)*10)/\x;
      } else {
        return 0.1 + 0.9/\x;
      };
    };
    function dcache(\x)
    {
      return lookup(1.0)/lookup(\x);
    };
    function rrcache(\x)
    {
      return lookup(1.0)/(0.1+0.9*(0.1 +(1-0.1)*10)/\x);
    };
    function pcache(\x)
    {
      return lookup(1.0)/(0.1+0.9/\x);
    };
    \a = dcache(1);
    \b = lookup(1.0);
  }
  \begin{tikzpicture}
    \begin{axis}[
      width=250pt,
      height=170pt,
      xlabel={\#threads},
      ylabel={Speedup},
      xlabel near ticks,
      ylabel near ticks,
      xmin=0,
      xmax=20,
      ymin=0,
      ymax=67,
      xtick={1,5,10,15,20},
      legend style = {
        anchor = north west,
        at = {(0.01, 1.01)},
        font=\scriptsize,
        % draw = none,
      },
      % no markers
      ]
      \addplot[domain=0:25,black,solid]{dcache(x)};
      \addlegendentry{Hash-based load balancing}
      \addplot[domain=0:25,black,densely dotted]{rrcache(x)};
      \addlegendentry{Random load balancing}
      \addplot[domain=0:25,black,densely dashed]{pcache(x)};
      \addlegendentry{All requests hit the cache}
      % \node at (25,25) {\a, \b};
    \end{axis}
  \end{tikzpicture}
\end{small}

%%% Local Variables:
%%% mode: latex
%%% TeX-master: "../distributed_mrf"
%%% End:

\end{small}
\caption{Scaling laws for distributed caching: hash-based load balancing, lower envelope (round robin load balancing) and upper envelope (perfect cache hit rate with $k$ caches). Parameters: $s=0.1$, $\delta=0.1$ and $\rho=10$.}
  \label{fig:dcache-analysis}
\end{figure}

% refer to "Modeling Speedup (n) Greater than n" -> analysis

% assumption for the analysis? ``I think equation 2 should be explained much better. It is not at all obvious (not sure even correct) that hit rate scales linearly with threads (I think only true is delta << 1). What is $\rho$ exactly the ratio of (fetching time in the event of miss) / (fetching time in the event of a catch hit)?''

It is instructive to quantify superlinear speedup in this context using a simple model. Suppose a source emits uniformly distributed random requests for $m$ items and requests are distributed among $k$ workers by hashing on the request id. This effectively partitions the requests into $k$ random buckets, so that each worker will perceive uniformly distributed requests for only $\frac{m}{k}$ items.  Suppose that, initially, the cache hit rate for a single worker that processes all $m$ possible requests is $\delta$. Then, scaling the system to $k$ workers will improve the cache hit rate to $k\delta$ ($k\delta \le 1$). This puts the lookup time of the system of $k$ parallel caches to
\begin{align}\label{eq:dist-cache}
  T_c(k) = \begin{cases} s + \frac{1-s}{k}(k\delta + (1-k\delta)\rho) & \text{if } k\delta \le 1\\s + \frac{(1-s)}{k} & \text{otherwise}\end{cases} \enspace ,
\end{align}
where $\rho$ is the penalty for a cache miss event.

The speedup $S_c(k)=\frac{T_c(1)}{T_c(k)}$ is depicted in Fig.\ref{fig:dcache-analysis}. The lower envelope of the scaling profile is given by Amdahl's law for the system with random or round robin load-balancing. % ($\frac{T_c(1)}{s + \frac{1-s}{k}(\delta + (1-\delta)\rho)}$).
As $k$ grows the scaling profile progresses over a superlinear curve to an elevated Amdahl's law profile, representative of a system where \emph{all} $\frac{m}{k}$ requests are served be workers from fast memory.  % ($\frac{T_c(1)}{s + \frac{1-s}{k}}$).
Note that this occurs \emph{only} if request dispatching is chosen carefully to partition the item space. For instance, modulo hashing achieves this but a random or a round robin load balancer does not, as these may send any item to any worker, which defeats the purpose of improving request locality at the workers' input. % and destroys superlinear scaling all together. % (see empirical evidence in the next Section).

Currently the only general methodology to achieve faster-than-linear scaling seems to be distributed caching, i.e., adding more fast memory to a system. Moving an application to a ``bigger machine'' \cite{dobb-2}, however, is not always feasible due to, e.g., physical or financial constraints.  In some cases caching cannot be used as it introduces more overhead than it saves, % due to some specifics of the problem domain,
e.g., for predominantly uniform input, rapidly changing data, inherently stateful computations, or complex database queries.  Moreover, caching comes with extra complexity \cite{271208}, including cache invalidation and eviction policies and data consistency mechanisms, which might be too costly to implement in a massively distributed setting. Apart from caching, however, currently the only way to achieve superlinearity is to rely on piecemeal problem-specific techniques, comprehensive domain knowledge, and pure luck. And even then, some compellingly argue that superlinear growth itself is a performance illusion, which goes against the very laws of nature much like perpetual motion \cite{gunther-hotsos, 10.1145/2773212.2789974}

% there is an actual conclusion in `` SUPERLINEAR SPEEDUP IN HPC SYSTEMS: WHY AND WHEN?``: ``Mainly the superlinear speedup performance in persistent algorithms occurs due to the increased cache re- sources in the parallel computer architectures, the prefetching of shared variables in shared memory organization, or better scheduling in heterogeneous environments.''

%%% Local Variables:
%%% mode: latex
%%% TeX-master: "distributed_mrf"
%%% End:

