\section{Case study }\label{sec:case-study}

Next, we present a case study for systematically applying the distributed self-adjusting systems architecture to a common networking problem: software packet classification \cite{gupta2001algorithms}. The goal is to demonstrate the general engineering methodology by assembling \emph{existing} techniques into a distributed self-adjusting scheme and understand when, and to what extent, superlinear scaling emerges. We consider it a success if we can robustly reproduce faster-than-linear growth on some realistic workloads. It is a stated \emph{nongoal} to conceive novel algorithms, let alone to produce the fastest existing software packet classifier. % (that award undeniably goes to DPDK \texttt{rte\_acl} \cite{rte-acl})
Yet, our self-adjusting firewall will prove several times faster than the default Linux kernel implementation on a wide range of workloads.

To achieve superlinear scaling we need a self-adjusting algorithm in the first place (plus a locality-boosting load balancer). From the many potential use cases % for which a self-adjusting algorithm exists
\cite{SleatorT85Splay, BentleyCL93, HesterH85, HesterH85, BentleySTW86, Avin0020, ParkM12} we eventually chose packet classification for the following reasons.  First, packet classifiers possess an intrinsic linear structure, % where the rules are matched in the order of priority for each packet,
which makes it an appealing candidate for applying the move-to-front (MTF) heuristics (but see ramifications related to handling rule-dependencies below). Second, the default Linux firewall implementation, \nftables, uses a static doubly linked list to match rules, which will buy us the non-self-adjusting baseline for free. Third, underlying packet classification there is an infamously difficult theoretical problem \cite{10.1145/2619239.2626294,10.1006/jagm.1996.0063, PacutVAPRS2022, 10.1145/2619239.2626294, 10.1145/1851182.1851208, 10.1145/863955.863980, gupta2001algorithms, 10.1145/3359989.3365431}, and achieving superlinear speedup on such a hard problem promises massive performance gain. Fourth, the Linux kernel network stack offers several flexible software and hardware based load balancers for dispatching packets to parallel classifier instances running on different CPU cores \cite{rss-linux}, which we will reuse to implement the locality-boosting load balancer component. And fifth, packet classifiers are very difficult to cache \cite{1354643} (recall, caches are the ``cheap'' way to obtain superlinear scaling), which calls for a true self-adjusting packet classifier algorithm that goes beyond caching. % \cite{10228937}.

\subsection{The Linux packet classier}
\label{sec:sa-pack-class}

A network firewall is a means to control incoming and outgoing network traffic based on user-defined packet classifier rules. This is useful to improve security, control access, filter and protect against ongoing attacks, and log\slash monitor network activity. The main concept in a classifier is a \emph{rule}, a pair of a filter and an action (see Fig.~\ref{fig:class-sample}). A filter is essentially a user-defined regular expression defined on specific fields of the packet header or metadata, and the action decides what to do with the packets that match the filter (accept, drop, log, etc.). Rules are organized into linear rule chains. When a packet enters a chain, it is compared against the first rule of the chain. If there is a match, the corresponding action is executed and the lookup is over. Otherwise, subsequent rules are matched in the order of priority until the first match is found.

The Linux kernel contains several built-in packet classifier implementations. The \nftables kernel engine adds a simple virtual machine to the Linux kernel that uses a DSL for parsing and matching packet header fields. This makes \nftables agnostic to specific network protocols, in contrast to, e.g., \texttt{iptables}, which contains an embedded protocol parser. Currently, \nftables is the default packet classifier in most Linux distributions.

\begin{table}[t]
  \centering
  \caption{Sample firewall rule set. Source ports do not matter.}
  \label{fig:class-sample}
  \begin{small}
    \renewcommand{\tabcolsep}{2pt}
    \begin{tabular}{r|l|l|r|r|l}
      \textbf{Prio} & \textbf{Proto} & \textbf{Src IP} & \textbf{Dst IP} & \textbf{Dst Port} & \textbf{Action}\\
      \hline
      1 & UDP & 192.168.178.33   & 23.0.0.45  & 53  & ACCEPT\\
      2 & TCP & 10.10.10.0/24    & 23.0.0.45  & 443 & DROP\\
      3 & UDP & 192.168.178.0/24 & 23.0.0.45  & 53  & DROP\\
      4 & TCP & 10.10.10.10/32   & 23.0.0.45  & ANY & ACCEPT\\
      5 & IP  & 192.168.0.0/16   & 23.0.0.0/8 &     & ACCEPT\\
    \end{tabular}
  \end{small}%
\end{table}

\subsection{Self-adjusting packet classification}
\label{sec:sa-sa-pack-class}

One way to make \nftables self-adjusting would be to replace the static linked list it uses internally for rule matching with a self-adjusting list. A naive application of MTF, however, would easily break the semantics of the firewall. This is because rules in the chain may not be independent from each other, and hence may not be freely swapped \cite{10.1145/2619239.2626294}.

Consider the example in Fig.~\ref{fig:class-sample} and suppose that, initially, rules are ordered priority-wise in the list: $\langle1, 2, 3, 4, 5\rangle$. Suppose that a packet with the IP 5-tuple (192.168.0.1, 23.0.0.45, UDP, 1, 3478) enters the classifier, where the fields in the 5-tuple are IP source and destination address, protocol, and source and destination port, respectively. Rules are inspected in linear order until rule 5 is found as the first match, at which point the lookup terminates with the verdict ACCEPT. Now, a naive application of MTF would move rule 5 to the front of list, resulting in the order $\langle5, 1, 2, 3, 4\rangle$. Suppose another packet with the 5-tuple (192.168.178.1, 23.0.0.45, UDP, 1, 53) is to be processed next: this will immediately match rule 5 at the front of the list yielding the verdict ACCEPT, despite that, if matched in priority order, rule 3 would be the correct match and the verdict should be DROP. To maintain correctness, the furthest we can move rule 5 towards the front of the list is the position immediately after its dependency, rule 3.

We say that rule $u$ is \emph{dependent} on another rule $v$ if they have overlapping match criteria in all fields, $v$ has a higher priority than $u$, and $u$ and $v$ define different actions. Such a dependency means that $u$ is not allowed to be moved before $v$ in the list, otherwise some packets may be erroneously classified. For instance, in the example of Fig.~\ref{fig:class-sample} rule 5 is dependent on rule 3, which is in turn dependent on rule 1, implying the dependency chain $5\to 3\to 1$. Similarly, rule $4$ is dependent on rule $2$. % Rule dependencies define a Directed Acyclic Graph (DAG) in the graph whose nodes are the set of rules, where there is an edge $(u, v)$ from $u$ to $v$ if $u$ is dependent on $v$ (see Fig.~\ref{fig:class-dep}).

% \begin{figure}[t]
%   \centering
%   \begin{small}
%     \begin{tikzpicture}[->,>=stealth,node distance=2.5cm, auto, every node/.style={circle,draw,minimum size=0.5cm,inner sep=2pt}]
%       % Branch 1
%       \node[circle,draw] (5) {5};
%       \node[circle,draw,right of=5] (3) {3};
%       \node[circle,draw,right of=3] (1) {1};

%       % Branch 2
%       \node[circle,draw,below of=5,yshift = 1.5cm] (4) {4};
%       \node[circle,draw,right of=4] (2) {2};

%       % Edges
%       \draw (5) -- (3);
%       \draw (3) -- (1);
%       \draw (4) -- (2);
%     \end{tikzpicture}
%   \end{small}
%   \caption{Dependency graph}%
%   \label{fig:class-dep}
% \end{figure}

A dependency-aware variant of the MTF heuristics, called the \emph{Move-recursively-Forward} (MRF) algorithm, is defined in \cite{10228937} (see Alg.~\ref{alg:mrf}). The idea is to push an accessed item forward in the list until the first dependency is reached. To prevent the item from blocking behind its direct dependency, the dependency is also moved forward until the first transitive dependency is hit. This process repeats until the head of the list is reached.  Independent rules are however free to be moved without restrictions, to the point that if there are no dependencies then MRF simplifies into a plain MTF policy.  Contrarily, if the entire rule set is a single dependency chain then no reordering is allowed and MRF degrades into a static list. In general, MRF moves frequently hit rules, with all their dependencies, to the first positions of the chain, which tends to improve linear lookup performance on high-locality data without altering the semantics of the classifier \cite{10228937}. In addition, MRF is ``almost''optimal in the strong competitive sense just like MTF, in that the best reordering one could obtain even if one knew the entire lookup sequence in advance would yield only a small constant factor improvement over MRF.

\begin{algorithm}[t]
  \caption{Move Recursively Forward (MRF)}
  \label{alg:mrf}
  \begin{small}
    \begin{algorithmic}[1]
      \Procedure{MRF}{$y$}
      \If{$y$ has no dependencies}
      \State Move $y$ to the front of the list
      \Else
      \State Let $z$ be the direct dependency of $y$
      \State Move node $y$ to position$(z) + 1$
      \State \Call{MRF}{$z$}
      \EndIf
      \EndProcedure
    \end{algorithmic}
  \end{small}
\end{algorithm}

Going back to our earlier example, after rule $5$ is hit in the list $\langle1, 2, 3, 4, 5\rangle$ MRF moves it immediately after the direct dependency $3$ along the dependency chain $5\to 3\to 1$, $3$ is moved to the position after $1$, and the recursion ends resulting the order $\langle1, 3, 2, 5, 4\rangle$. If $5$ was hit again, the lookup time would be only $4$ instead of $5$. Then, $5$ would be moved forward again, yielding the order $\langle1, 3, 5, 2, 4\rangle$ and a lookup time of $3$. Note that dependency chains can be moved by MRF independently from each other: e.g., if $4$ was hit first then it would be moved after its dependency $2$, yielding $\langle2, 1, 4, 3, 5\rangle$ in the first iteration and eventually $\langle2, 4, 1, 3, 5\rangle$, with lookup time for $4$ dropping to $2$.

\subsection{Locality-boosting load balancing}
\label{sec:sa-rss}

The above discussion suggests that an ideal locality-boosting load balancer would partition the rule set dependency-wise, keeping independent rule dependency chains at different workers. Since such independent chains are allowed to be moved around freely, MRF at each worker would eventually arrange the worker's active rules, with their dependencies, to the first positions, obtaining a faster lookup time. Unfortunately, it is difficult to obtain such ``perfectly partitioning'' load balancers in practice. Therefore, we resort to different imperfect approximations of the ideal policy in our implementation.

The difficulty lies in that packet classifiers use wildcard rules, each matching potentially a huge number of different packets, and hash functions may assign those packets to different workers. Once two packets matching the same rule appear at two different workers, both workers will have to add the same rule, with all its dependencies, to its active rule set, and try to move it towards the front. This imperfect partitioning tends to blow up active rule sets at the workers, deteriorating lookup performance. The same rule duplication problem plagues many software packet classifier algorithms and substantial active research activity focuses on finding ways to prevent it \cite{10.1145/863955.863980, 820051, 10.1145/1851182.1851208, 8485947}. Instead of adopting the ideas from recent research, however, we will opt for a simple hash-based load balancing scheme below, since this is very easy to implement in common NIC hardware. Later, we will show empirically that even this imperfect scheme is enough to reach superlinear speedup in many practical cases.

The simplest load balancer for scaling the Linux networking stack to multiple CPU cores is Receive Side Scaling (RSS, \cite{10.1145/3359989.3365412, rss-linux}). Most NICs support multiple packet queues to distribute packets to different CPUs. In order to preserve processing order for the different packets of logical flows, NICs typically apply a hash function over the packet header fields. The resultant hash value is then used as an index into an indirection table to select the queue, and hence the CPU core, that will process the packet. Since a ``logical flow'' can mean a different thing in different contexts, the hash function is configurable to consider only certain header fields, e.g., only the source IP address (``sticky sessions'') or the entire IP 5-tuple.

\subsection{Implementation}
\label{sec:sa-nf-tables-impl}

We created a comprehensive self-adjusting packet classifier implementation on top of \nftables. Our implementation uses the dependency-aware MRF algorithm to match rules from a self-adjusting list \cite{10228937}. Each classifier instance maintains a per-CPU array of pointers that index into a shared static rule list, running the MRF algorithm on its pointer array instead of directly on the rule list. This allows for running as many parallel MRF instances as there are CPUs without lock contention, with each MRF instance maintaining its own rule order. We can also support lockless rule addition\slash deletion this way: every time the rule list is updated we simply allocate a new pointer array at each CPU and update the list head atomically.

The original MRF algorithm \cite{10228937} uses recursion (see Alg.~\ref{alg:mrf}). Recursion, however, is expensive in the Linux kernel due to the overhead of maintaining the function call stack. To avoid this overhead, we defined an iterative version of the MRF algorithm. When a rule is to be moved forward, we first check whether it can be swapped with the preceding rule. This is done by checking whether the two rules overlap using a range-based representation, which we extract from the rule's bytecode in the \nftables virtual machine. If there is an overlap then the rule cannot be moved further so we restart the process, trying to move the blocking dependency forward. Otherwise, the two rules are independent, so they are immediately swapped and the iteration is repeated with the subsequent preceding rule. The iteration terminates when we reach the first position. A more efficient implementation would be to precompute dependencies on rule insertion\slash deletion and run the recursive MRF algorithm using the cached dependencies. % This, however, would complicate code, make insertions more expensive, and may even end up being slower since recursion in the kernel can be costly due to the overhead of a potentially deep call stack.

\subsection{Evaluation}
\label{sec:sa-nf-tables-eval}

We conducted several experiments with the distributed self-adjusting packet classifier, using a simple hash-based RSS load balancer to distribute load. Our goal was to understand whether superlinear scaling can indeed be reproduced on a real network application and using real packet I/O and determine the conditions (rule sets, flow size, etc) under which it emerges.

\noindent
\textbf{Testbed.} %

\begin{itemize}
\item server specs
\item tipsy~\cite{8468219}
\end{itemize}

\noindent
\textbf{Macrobenchmarks.} %
\begin{itemize}
\item classbench: speedup and raw packet rate with latency in inset for 3 classbench seeds: acl, ipc and firewall
\item takeaway: superlinear scaling emerges in real workloads + baseline (nftables) only linearly scales + SA version is much faster + imperfect hash-based LB works + dependency matters (fw seed result is poor)
\end{itemize}

\begin{figure*}
  \centering
  \subfloat[][acl1]{
    \resizebox{.225\textwidth}{!}{
      \pgfplotsset{
  RatePlot/.style = {
    tick pos = left,
    xtick align=outside,
    ytick align=outside,
    xlabel near ticks,
    ylabel near ticks,
    width=.4\textwidth,
    height=.3\textwidth,
    legend pos = north west,
    legend cell align=left,
    ylabel = {Throughput [Mpps]},
    xlabel = {number of CPU cores},
    xmin=1, xmax=32,
    ymin=0,
  },
  SpeedupPlot/.style = {
    RatePlot,
    ylabel={Speedup},
  },
  ClassBenchGroupPlot/.style = {
    group/group size = 1 by 2,
    group/horizontal sep = 0pt,
    group/vertical sep = 28pt,
  },
  ClassBenchRatePlot/.style = {
    RatePlot
  },
  ClassBenchSpeedupPlot/.style = {
    SpeedupPlot,
  }
}

%%% Local Variables:
%%% mode: latex
%%% TeX-master: "../distributed_mrf.tex"
%%% End:


\begin{tikzpicture}

 \begin{groupplot}[ClassBenchGroupPlot]

  \nextgroupplot[ClassBenchSpeedupPlot]

  \addplot coordinates {
     (1, 1.00000)
     (2, 2.01275)
     (3, 2.99326)
     (4, 4.03999)
     (5, 5.02312)
     (6, 6.02253)
     (7, 7.01021)
     (8, 8.04299)
     (9, 9.02064)
     (10, 10.03534)
     (11, 11.10694)
     (12, 12.08788)
     (13, 13.07358)
     (14, 14.01179)
     (15, 15.05161)
     (16, 16.13280)
     (17, 17.14857)
     (18, 18.07133)
     (19, 19.04549)
     (20, 20.20622)
     (21, 21.20803)
     (22, 22.09785)
     (23, 23.04393)
     (24, 24.23765)
     (25, 25.17218)
     (26, 26.28756)
     (27, 27.06453)
     (28, 27.74060)
     (29, 28.61572)
     (30, 29.48150)
     (31, 30.28735)
     (32, 31.37051)
     (33, 31.31105)
     (34, 31.46892)
     (35, 31.41659)
     (36, 31.50497)
     (37, 31.88759)
     (38, 31.98595)
     (39, 32.18120)
     (40, 32.26755)
     (41, 32.51116)
     (42, 32.68041)
     (43, 33.08639)
     (44, 33.06703)
     (45, 33.14436)
     (46, 33.23806)
     (47, 33.46593)
     (48, 33.40066)
     (49, 33.61066)
     (50, 33.73653)
     (51, 34.05734)
     (52, 34.13124)
     (53, 34.31725)
     (54, 34.47786)
     (55, 34.61957)
     (56, 34.64922)
     (57, 34.57655)
     (58, 34.92709)
     (59, 34.83238)
     (60, 35.29259)
     (61, 35.46089)
     (62, 35.39285)
     (63, 35.38630)
    };
    \addlegendentry{Static}

  \addplot coordinates {
     (1, 1.00000)
     (2, 2.78880)
     (3, 5.34148)
     (4, 8.50314)
     (5, 12.18752)
     (6, 16.29228)
     (7, 20.91828)
     (8, 25.42996)
     (9, 30.76640)
     (10, 35.64435)
     (11, 42.18526)
     (12, 47.99208)
     (13, 54.94981)
     (14, 61.30928)
     (15, 67.57976)
     (16, 76.48709)
     (17, 84.80360)
     (18, 88.83299)
     (19, 98.87346)
     (20, 105.97422)
     (21, 117.95420)
     (22, 124.30595)
     (23, 130.33294)
     (24, 141.99362)
     (25, 151.39788)
     (26, 164.00849)
     (27, 169.80391)
     (28, 181.86091)
     (29, 191.72184)
     (30, 196.66672)
     (31, 210.59534)
     (32, 225.16466)
     (33, 224.40485)
     (34, 229.31071)
     (35, 229.13107)
     (36, 236.10585)
     (37, 249.26799)
     (38, 243.55203)
     (39, 246.00255)
     (40, 256.27063)
     (41, 260.25177)
     (42, 266.14283)
     (43, 269.95816)
     (44, 277.40640)
     (45, 281.26218)
     (46, 280.08708)
     (47, 285.01959)
     (48, 289.29034)
     (49, 290.48724)
     (50, 294.05818)
     (51, 298.84372)
     (52, 304.31674)
     (53, 306.37411)
     (54, 305.79708)
     (55, 306.30851)
     (56, 309.33107)
     (57, 313.83119)
     (58, 320.11412)
     (59, 329.67394)
     (60, 323.96154)
     (61, 336.04423)
     (62, 334.62710)
     (63, 342.45456)
    };
    \addlegendentry{Self-adjusting}


  \nextgroupplot[ClassBenchRatePlot]

    \addplot coordinates {
      (1, 0.046539629713245)
      (2, 0.093672584131231)
      (3, 0.1393051523559)
      (4, 0.18801951171016)
      (5, 0.2337741675624)
      (6, 0.28028621268916)
      (7, 0.32625249458415)
      (8, 0.37431778626937)
      (9, 0.41981712674747)
      (10, 0.46704104588755)
      (11, 0.51691309033259)
      (12, 0.56256543220967)
      (13, 0.60843936704226)
      (14, 0.65210352613963)
      (15, 0.70049618756688)
      (16, 0.75081434963534)
      (17, 0.79808802554201)
      (18, 0.84103282307131)
      (19, 0.8863702768161)
      (20, 0.9403901302883)
      (21, 0.98701389451984)
      (22, 1.0284257030863)
      (23, 1.0724561987193)
      (24, 1.1280112041682)
      (25, 1.1715041448177)
      (26, 1.2234130913758)
      (27, 1.2595730640873)
      (28, 1.2910370796756)
      (29, 1.3317647863544)
      (30, 1.3720579499166)
      (31, 1.4095619991636)
      (32, 1.4599720329117)
      (33, 1.4572048750005)
      (34, 1.4645519240141)
      (35, 1.4621164290821)
      (36, 1.4662297251988)
      (37, 1.4840364875665)
      (38, 1.4886142627512)
      (39, 1.4977010671677)
      (40, 1.5017199796616)
      (41, 1.513057464791)
      (42, 1.5209341329062)
      (43, 1.5398284158632)
      (44, 1.5389273111143)
      (45, 1.5425261198347)
      (46, 1.5468869141326)
      (47, 1.5574918678777)
      (48, 1.5544544573799)
      (49, 1.5642275725289)
      (50, 1.570085466313)
      (51, 1.5850162120482)
      (52, 1.5884550498825)
      (53, 1.5971119199612)
      (54, 1.6045867927364)
      (55, 1.6111820668819)
      (56, 1.6125620968556)
      (57, 1.6091796950591)
      (58, 1.625494043245)
      (59, 1.6210858597396)
      (60, 1.6425038639144)
      (61, 1.6503368924279)
      (62, 1.6471702537073)
      (63, 1.646865499047)
    };
    %\addlegendentry{Static}

    \addplot coordinates {
      (1, 0.032711671064758)
      (2, 0.091226367386569)
      (3, 0.17472861741391)
      (4, 0.27815189405222)
      (5, 0.39867428595652)
      (6, 0.53294756137404)
      (7, 0.68427196615884)
      (8, 0.83185664878198)
      (9, 1.0064203293998)
      (10, 1.1659863048086)
      (11, 1.3799504801709)
      (12, 1.5699010587605)
      (13, 1.7975000467864)
      (14, 2.0055291591319)
      (15, 2.2106470134093)
      (16, 2.5020206145042)
      (17, 2.7740673059055)
      (18, 2.9058754936516)
      (19, 3.2343161023274)
      (20, 3.4665937520818)
      (21, 3.8584788645415)
      (22, 4.0662555102297)
      (23, 4.2634082896897)
      (24, 4.6448485014523)
      (25, 4.9524775583613)
      (26, 5.364991630779)
      (27, 5.5545694871632)
      (28, 5.9489741357534)
      (29, 6.271541634084)
      (30, 6.433297006678)
      (31, 6.8889255847637)
      (32, 7.3655123135189)
      (33, 7.340657797016)
      (34, 7.5011366031331)
      (35, 7.4952600759246)
      (36, 7.7234167470822)
      (37, 8.1539723522886)
      (38, 7.9669939347613)
      (39, 8.0471546196194)
      (40, 8.3830405627332)
      (41, 8.5132704352016)
      (42, 8.7059768571641)
      (43, 8.8307824941431)
      (44, 9.0744269109974)
      (45, 9.2005560451181)
      (46, 9.162116552901)
      (47, 9.3234670053344)
      (48, 9.4631704141204)
      (49, 9.5023229923151)
      (50, 9.6191344074074)
      (51, 9.7756773452302)
      (52, 9.954709011936)
      (53, 10.022009206075)
      (54, 10.00313342368)
      (55, 10.019863323224)
      (56, 10.118736180131)
      (57, 10.265942613862)
      (58, 10.471467947096)
      (59, 10.784185365095)
      (60, 10.597323449753)
      (61, 10.992568407591)
      (62, 10.946211746648)
      (63, 11.20226094469)
    };
    %\addlegendentry{Self-adjusting}

  \end{groupplot}

  % Inset delay figure
  %\begin{axis}[
  RatePlot,
  xshift = 2.5pt,
  yshift = -32.5pt,
  width=87.5pt,
  height=65pt,
  ylabel shift = -5 pt,
  xlabel shift = -5 pt,
  xlabel = {\# CPU cores},
  ylabel = {Delay [ms]},
  major tick length = 1.5,
  minor tick length = .75,
  xtick align = inside,
  ytick align = inside,
  ytick pos = right,
  xlabel shift = -5pt,
  xticklabel shift = -2pt,
  ylabel shift = -5pt,
  yticklabel shift = -2pt,
  yticklabel pos = right,
  ticklabel style = {font = \tiny},
  label style = {font = \tiny},
  font = \tiny,
  ]

  \addplot[StaticMarkSmall] coordinates {
    (1, 11.5246859)
    (2, 13.0414406)
    (3, 12.0516008)
    (4, 12.4346342)
    (5, 12.7136874)
    (6, 12.4196939)
    (7, 12.9454364)
    (8, 12.5504689)
    (9, 13.1494149)
    (10, 13.4371564)
    (11, 13.2305652)
    (12, 13.4584262)
    (13, 13.6475760)
    (14, 13.9229871)
    (15, 13.6530165)
    (16, 13.5959664)
    (17, 13.7759351)
    (18, 13.9005972)
    (19, 13.8574247)
    (20, 13.9209600)
    (21, 14.0194501)
    (22, 14.0310233)
    (23, 14.1952436)
    (24, 13.8615738)
    (25, 13.7690495)
    (26, 13.6904902)
    (27, 13.9852999)
    (28, 14.0805306)
    (29, 14.0971204)
    (30, 14.0061332)
    (31, 13.6011743)
    (32, 13.4245527)
    (33, 14.0842828)
    (34, 14.1634843)
    (35, 14.4694397)
    (36, 14.2107712)
    (37, 14.6166717)
    (38, 14.4130623)
    (39, 14.3040049)
    (40, 13.4596624)
    (41, 14.0578499)
    (42, 14.2140139)
    (43, 14.4488030)
    (44, 14.3561524)
    (45, 14.2936769)
    (46, 14.4158717)
    (47, 14.6920556)
    (48, 14.4894246)
    (49, 14.5674845)
    (50, 14.6061373)
    (51, 14.2901432)
    (52, 14.3238256)
    (53, 14.3649980)
    (54, 14.3366655)
    (55, 14.5369478)
    (56, 14.8161641)
    (57, 14.6632713)
    (58, nan)
    (59, 14.6380668)
    (60, 13.9192529)
    (61, 13.9991509)
    (62, 13.7683118)
    (63, nan)
  };

  \addplot[SelfAdjustingMarkSmall] coordinates {
    (1, 13.7504560)
    (2, 12.8632773)
    (3, 10.5175126)
    (4, 10.3660933)
    (5, 09.0034255)
    (6, 07.9318335)
    (7, 07.8026594)
    (8, 06.3111752)
    (9, 06.5644485)
    (10, 06.2939857)
    (11, 05.5898275)
    (12, 05.3072441)
    (13, 05.6942094)
    (14, 04.9724582)
    (15, 04.1699062)
    (16, 03.9506734)
    (17, 04.2641513)
    (18, 04.4839685)
    (19, 03.7867425)
    (20, 03.8697160)
    (21, 03.6709087)
    (22, 03.5707060)
    (23, 03.7592740)
    (24, 03.7324230)
    (25, 03.0875692)
    (26, 03.4737530)
    (27, 02.6244803)
    (28, 02.5924367)
    (29, 02.2520316)
    (30, 02.9590999)
    (31, 02.1526739)
    (32, 02.1813730)
    (33, 02.5701373)
    (34, 02.7773967)
    (35, 03.8287018)
    (36, 03.0827487)
    (37, 02.4387644)
    (38, 02.6892771)
    (39, 02.8402511)
    (40, 02.8046372)
    (41, 02.3915482)
    (42, 02.0204283)
    (43, 02.6570434)
    (44, 02.9146742)
    (45, 02.6903529)
    (46, 04.1966236)
    (47, 03.9124022)
    (48, 03.1979138)
    (49, 01.8601223)
    (50, 0.31112418)
    (51, 0.19742585)
    (52, 0.16013978)
    (53, 0.07419928)
    (54, 0.20887176)
    (55, 0.12476317)
    (56, 0.12985358)
    (57, 0.20131270)
    (58, 0.09126601)
    (59, 0.24982249)
    (60, 0.16912486)
    (61, 0.12356464)
    (62, 0.06734899)
    (63, 0.28888528)
  };
\end{axis}


\end{tikzpicture}

      \label{fig:classbench-acl1}
    }
  }
  \subfloat[][ipc1]{
    \resizebox{.225\textwidth}{!}{
      \pgfplotsset{
  RatePlot/.style = {
    tick pos = left,
    xtick align=outside,
    ytick align=outside,
    xlabel near ticks,
    ylabel near ticks,
    width=.4\textwidth,
    height=.3\textwidth,
    legend pos = north west,
    legend cell align=left,
    ylabel = {Throughput [Mpps]},
    xlabel = {number of CPU cores},
    xmin=1, xmax=32,
    ymin=0,
  },
  SpeedupPlot/.style = {
    RatePlot,
    ylabel={Speedup},
  },
  ClassBenchGroupPlot/.style = {
    group/group size = 1 by 2,
    group/horizontal sep = 0pt,
    group/vertical sep = 28pt,
  },
  ClassBenchRatePlot/.style = {
    RatePlot
  },
  ClassBenchSpeedupPlot/.style = {
    SpeedupPlot,
  }
}

%%% Local Variables:
%%% mode: latex
%%% TeX-master: "../distributed_mrf.tex"
%%% End:


\begin{tikzpicture}

 \begin{groupplot}[ClassBenchGroupPlot]

  \nextgroupplot[ClassBenchSpeedupPlot]

  \addplot coordinates {
     (1, 1.00000)
     (2, 1.99904)
     (3, 2.89529)
     (4, 3.94680)
     (5, 4.94148)
     (6, 5.74940)
     (7, 6.84421)
     (8, 7.87120)
     (9, 8.68609)
     (10, 9.62543)
     (11, 10.76151)
     (12, 11.59969)
     (13, 12.48071)
     (14, 13.37427)
     (15, 14.31106)
     (16, 15.34732)
     (17, 16.21912)
     (18, 17.26689)
     (19, 18.25517)
     (20, 18.99125)
     (21, 20.10330)
     (22, 21.12825)
     (23, 21.98638)
     (24, 23.02527)
     (25, 23.99193)
     (26, 25.00409)
     (27, 25.96916)
     (28, 26.76306)
     (29, 27.67258)
     (30, 28.77140)
     (31, 29.50573)
     (32, 30.37490)
     (33, 30.40629)
     (34, 30.64074)
     (35, 30.72365)
     (36, 31.37580)
     (37, 31.42249)
     (38, 31.81717)
     (39, 32.17639)
     (40, 32.28154)
     (41, 32.51125)
     (42, 32.95661)
     (43, 33.02789)
     (44, 33.48791)
     (45, 34.04567)
     (46, 34.06134)
     (47, 34.66463)
     (48, 34.72688)
     (49, 35.10103)
     (50, 35.47690)
     (51, 35.60816)
     (52, 35.97768)
     (53, 36.07897)
     (54, 36.55234)
     (55, 36.74071)
     (56, 37.00593)
     (57, 37.28585)
     (58, 37.56701)
     (59, 37.85790)
     (60, 38.13999)
     (61, 38.65728)
     (62, 38.72069)
     (63, 38.99848)
    };
    \addlegendentry{Static}

  \addplot coordinates {
     (1, 1.00000)
     (2, 2.20256)
     (3, 3.66848)
     (4, 5.28803)
     (5, 7.07860)
     (6, 9.02878)
     (7, 10.92263)
     (8, 12.96272)
     (9, 15.22643)
     (10, 17.22539)
     (11, 19.33919)
     (12, 21.74296)
     (13, 24.06457)
     (14, 26.16635)
     (15, 28.95934)
     (16, 30.80881)
     (17, 33.36339)
     (18, 35.92651)
     (19, 37.90040)
     (20, 40.60494)
     (21, 43.38436)
     (22, 46.13333)
     (23, 48.23023)
     (24, 51.33538)
     (25, 53.98739)
     (26, 56.26039)
     (27, 59.36582)
     (28, 61.28844)
     (29, 65.46223)
     (30, 67.96489)
     (31, 69.39216)
     (32, 72.02270)
     (33, 73.73031)
     (34, 74.33798)
     (35, 75.79911)
     (36, 78.07703)
     (37, 78.74681)
     (38, 78.92463)
     (39, 83.23712)
     (40, 81.81416)
     (41, 83.07275)
     (42, 85.40033)
     (43, 87.80475)
     (44, 89.72449)
     (45, 89.83877)
     (46, 89.94273)
     (47, 92.81415)
     (48, 93.06880)
     (49, 95.82243)
     (50, 96.56137)
     (51, 98.81827)
     (52, 98.02214)
     (53, 102.84065)
     (54, 103.14348)
     (55, 102.71015)
     (56, 103.91244)
     (57, 103.91456)
     (58, 110.49607)
     (59, 112.25616)
     (60, 111.94357)
     (61, 115.83462)
     (62, 114.89240)
     (63, 115.08863)
    };
    \addlegendentry{Self-adjusting}


  \nextgroupplot[ClassBenchRatePlot]

    \addplot coordinates {
      (1, 0.031744111665511)
      (2, 0.06345764248803)
      (3, 0.091908519541636)
      (4, 0.12528766365938)
      (5, 0.15686304594011)
      (6, 0.18250956925555)
      (7, 0.21726343486567)
      (8, 0.24986432095669)
      (9, 0.27573229304931)
      (10, 0.30555056652382)
      (11, 0.34161460030774)
      (12, 0.36822186805517)
      (13, 0.39618898479011)
      (14, 0.42455421455473)
      (15, 0.45429173415525)
      (16, 0.48718705524847)
      (17, 0.51486159839532)
      (18, 0.54812209288819)
      (19, 0.57949418947039)
      (20, 0.60286035824686)
      (21, 0.63816150612146)
      (22, 0.67069744081536)
      (23, 0.69793805530304)
      (24, 0.73091662482903)
      (25, 0.76160264715354)
      (26, 0.79373269598525)
      (27, 0.82436782385871)
      (28, 0.84956963537846)
      (29, 0.87844154226898)
      (30, 0.91332255008854)
      (31, 0.93663321831887)
      (32, 0.96422433146812)
      (33, 0.96522063738741)
      (34, 0.97266295711053)
      (35, 0.97529501211155)
      (36, 0.99599704950763)
      (37, 0.9974789718364)
      (38, 1.0100078956506)
      (39, 1.0214108255887)
      (40, 1.0247489335771)
      (41, 1.0320407243525)
      (42, 1.0461783410233)
      (43, 1.0484410263942)
      (44, 1.0630439417362)
      (45, 1.0807493982743)
      (46, 1.0812469142839)
      (47, 1.1003979788613)
      (48, 1.1023740588795)
      (49, 1.114251025749)
      (50, 1.1261827795326)
      (51, 1.1303495040476)
      (52, 1.1420795825023)
      (53, 1.1452948199886)
      (54, 1.1603215255264)
      (55, 1.1663011271238)
      (56, 1.1747203713183)
      (57, 1.1836061503951)
      (58, 1.1925312282163)
      (59, 1.2017655298065)
      (60, 1.2107201289644)
      (61, 1.2271411601153)
      (62, 1.2291540532193)
      (63, 1.2379719857676)
    };
    %\addlegendentry{Static}

    \addplot coordinates {
      (1, 0.018889987135068)
      (2, 0.041606239750658)
      (3, 0.069297553101307)
      (4, 0.09989082130905)
      (5, 0.13371469324804)
      (6, 0.17055354254507)
      (7, 0.20632828893924)
      (8, 0.24486570461986)
      (9, 0.28762697309928)
      (10, 0.32538748583933)
      (11, 0.36531712499108)
      (12, 0.41072423416958)
      (13, 0.45457945262613)
      (14, 0.49428201504977)
      (15, 0.54704156833915)
      (16, 0.58197799770858)
      (17, 0.63023392021364)
      (18, 0.67865132429634)
      (19, 0.71593800237838)
      (20, 0.76702670262749)
      (21, 0.81953000032308)
      (22, 0.87145801355135)
      (23, 0.91106846222972)
      (24, 0.96972458688005)
      (25, 1.0198211188854)
      (26, 1.0627580921815)
      (27, 1.1214195184789)
      (28, 1.1577377491515)
      (29, 1.2365806896474)
      (30, 1.2838558959822)
      (31, 1.3108169705898)
      (32, 1.3605078081155)
      (33, 1.3927646148772)
      (34, 1.4042435625386)
      (35, 1.4318442480682)
      (36, 1.4748740145751)
      (37, 1.4875261749455)
      (38, 1.4908852601269)
      (39, 1.5723481148306)
      (40, 1.545468447653)
      (41, 1.5692432165363)
      (42, 1.6132111202603)
      (43, 1.6586305311713)
      (44, 1.6948944717015)
      (45, 1.6970531644274)
      (46, 1.6990170088985)
      (47, 1.753258037858)
      (48, 1.7580683551754)
      (49, 1.810084485413)
      (50, 1.8240430490041)
      (51, 1.8666759363908)
      (52, 1.8516369793844)
      (53, 1.9426585570425)
      (54, 1.9483789317255)
      (55, 1.9401934934174)
      (56, 1.9629046074725)
      (57, 1.9629446910417)
      (58, 2.0872693483339)
      (59, 2.1205175033934)
      (60, 2.1146125661818)
      (61, 2.1881144974859)
      (62, 2.1703159284934)
      (63, 2.1740227160904)
    };
    %\addlegendentry{Self-adjusting}

  \end{groupplot}
\end{tikzpicture}

      \label{fig:classbench-ipc1}
    }
  }
  \subfloat[][fw1]{
    \resizebox{.225\textwidth}{!}{
      \pgfplotsset{
  RatePlot/.style = {
    tick pos = left,
    xtick align=outside,
    ytick align=outside,
    xlabel near ticks,
    ylabel near ticks,
    width=.4\textwidth,
    height=.3\textwidth,
    legend pos = north west,
    legend cell align=left,
    ylabel = {Throughput [Mpps]},
    xlabel = {number of CPU cores},
    xmin=1, xmax=32,
    ymin=0,
  },
  SpeedupPlot/.style = {
    RatePlot,
    ylabel={Speedup},
  },
  ClassBenchGroupPlot/.style = {
    group/group size = 1 by 2,
    group/horizontal sep = 0pt,
    group/vertical sep = 28pt,
  },
  ClassBenchRatePlot/.style = {
    RatePlot
  },
  ClassBenchSpeedupPlot/.style = {
    SpeedupPlot,
  }
}

%%% Local Variables:
%%% mode: latex
%%% TeX-master: "../distributed_mrf.tex"
%%% End:


\begin{tikzpicture}

 \begin{groupplot}[ClassBenchGroupPlot]

  \nextgroupplot[ClassBenchSpeedupPlot, ytick distance={25},]

  \addplot[StaticMark,each nth point={2}] coordinates {
     (1, 1.00000)
     (2, 2.00642)
     (3, 3.00227)
     (4, 3.98526)
     (5, 4.97784)
     (6, 5.97126)
     (7, 6.95429)
     (8, 7.92904)
     (9, 8.95196)
     (10, 9.95693)
     (11, 10.95708)
     (12, 11.90737)
     (13, 12.98807)
     (14, 13.89798)
     (15, 14.90286)
     (16, 15.90926)
     (17, 16.87838)
     (18, 17.88937)
     (19, 18.83991)
     (20, 19.79596)
     (21, 20.79066)
     (22, 21.78117)
     (23, 22.95279)
     (24, 23.92195)
     (25, 24.87083)
     (26, 25.97827)
     (27, 26.83614)
     (28, 27.48385)
     (29, 28.32628)
     (30, 29.25493)
     (31, 30.03297)
     (32, 30.49489)
     (33, 30.87020)
     (34, 31.08683)
     (35, 31.13049)
     (36, 31.37795)
     (37, 31.55764)
     (38, 31.83831)
     (39, 31.74457)
     (40, 31.98458)
     (41, 32.29963)
     (42, 32.32307)
     (43, 32.32149)
     (44, 32.63861)
     (45, 32.67067)
     (46, 32.75227)
     (47, 32.93674)
     (48, 32.95525)
     (49, 33.01142)
     (50, 33.21486)
     (51, 33.49748)
     (52, 33.71485)
     (53, 33.94554)
     (54, 33.82370)
     (55, 34.05714)
     (56, 34.42454)
     (57, 34.30365)
     (58, 34.57126)
     (59, 34.62224)
     (60, 34.84287)
     (61, 34.92018)
     (62, 35.08595)
     (63, 35.27453)
    };
    \addlegendentry{Static}

  \addplot[SelfAdjustingMark,each nth point={2}] coordinates {
     (1, 1.00000)
     (2, 2.10620)
     (3, 3.32994)
     (4, 4.60646)
     (5, 6.04220)
     (6, 7.52206)
     (7, 8.85721)
     (8, 10.44222)
     (9, 11.97781)
     (10, 13.58400)
     (11, 15.12528)
     (12, 16.83646)
     (13, 18.62258)
     (14, 19.90959)
     (15, 22.03945)
     (16, 23.29472)
     (17, 25.05463)
     (18, 27.26867)
     (19, 28.36004)
     (20, 30.56113)
     (21, 32.33906)
     (22, 33.74235)
     (23, 35.46088)
     (24, 37.95365)
     (25, 39.50008)
     (26, 42.13974)
     (27, 43.99437)
     (28, 45.11863)
     (29, 47.74894)
     (30, 50.61731)
     (31, 51.52214)
     (32, 52.83959)
     (33, 53.56519)
     (34, 54.73500)
     (35, 56.08416)
     (36, 56.88490)
     (37, 58.82567)
     (38, 57.26452)
     (39, 61.12304)
     (40, 60.10926)
     (41, 61.54647)
     (42, 60.44772)
     (43, 65.14732)
     (44, 62.42518)
     (45, 64.11267)
     (46, 64.33649)
     (47, 65.25661)
     (48, 66.65516)
     (49, 67.07342)
     (50, 69.85434)
     (51, 68.17294)
     (52, 70.97795)
     (53, 71.39927)
     (54, 74.81633)
     (55, 74.06910)
     (56, 75.78303)
     (57, 73.92442)
     (58, 78.45669)
     (59, 78.68256)
     (60, 81.34688)
     (61, 79.42674)
     (62, 82.14868)
     (63, 81.65074)
    };
    \addlegendentry{Self-adjusting}


    \nextgroupplot[ClassBenchRatePlot, ytick distance={.25},]

    \addplot[StaticMark,each nth point={2}] coordinates {
      (1, 0.046175337275456)
      (2, 0.09264702209592)
      (3, 0.13863104867912)
      (4, 0.18402089749063)
      (5, 0.22985352206762)
      (6, 0.27572504514474)
      (7, 0.32111657076421)
      (8, 0.36612603036957)
      (9, 0.41335992529034)
      (10, 0.45976448979103)
      (11, 0.50594685313329)
      (12, 0.54982685620667)
      (13, 0.59972869147124)
      (14, 0.64174406455464)
      (15, 0.68814442112783)
      (16, 0.73461527598681)
      (17, 0.77936467117974)
      (18, 0.82604754443381)
      (19, 0.86993899027163)
      (20, 0.91408535287356)
      (21, 0.96001563210835)
      (22, 1.0057528255294)
      (23, 1.0598530159052)
      (24, 1.1046042462612)
      (25, 1.1484188298822)
      (26, 1.1995551986021)
      (27, 1.2391676499915)
      (28, 1.2690762528259)
      (29, 1.3079755350045)
      (30, 1.3508561030092)
      (31, 1.3867826333745)
      (32, 1.4081119335249)
      (33, 1.4254417984352)
      (34, 1.4354447905783)
      (35, 1.4374608756753)
      (36, 1.4488872783868)
      (37, 1.457184487247)
      (38, 1.4701448721321)
      (39, 1.4658163675455)
      (40, 1.4768986840629)
      (41, 1.4914464990695)
      (42, 1.4925285957533)
      (43, 1.4924556002395)
      (44, 1.5070988035406)
      (45, 1.5085793445279)
      (46, 1.5123468886761)
      (47, 1.5208652043167)
      (48, 1.5217199585225)
      (49, 1.5243135939772)
      (50, 1.5337073923782)
      (51, 1.5467576297889)
      (52, 1.5567944652189)
      (53, 1.5674465598604)
      (54, 1.561820787017)
      (55, 1.5725999883676)
      (56, 1.5895646786313)
      (57, 1.5839824554285)
      (58, 1.5963394180261)
      (59, 1.5986934308387)
      (60, 1.6088812904173)
      (61, 1.6124509802072)
      (62, 1.6201056318048)
      (63, 1.6288131272854)
    };
    %\addlegendentry{Static}

    \addplot[SelfAdjustingMark,each nth point={2}] coordinates {
      (1, 0.026644006858161)
      (2, 0.056117557869066)
      (3, 0.088723041020338)
      (4, 0.12273458673186)
      (5, 0.16098834116769)
      (6, 0.20041792048914)
      (7, 0.23599149569214)
      (8, 0.27822253191441)
      (9, 0.31913689382905)
      (10, 0.36193211309214)
      (11, 0.40299813009226)
      (12, 0.44859084774872)
      (13, 0.49618009062294)
      (14, 0.53047118514086)
      (15, 0.58721926451141)
      (16, 0.62066461989621)
      (17, 0.66755582214289)
      (18, 0.72654675580445)
      (19, 0.75562501559322)
      (20, 0.81427091549728)
      (21, 0.86164203247128)
      (22, 0.89903130126662)
      (23, 0.94481992091986)
      (24, 1.0112373794178)
      (25, 1.0524403251209)
      (26, 1.1227715208184)
      (27, 1.1721861953998)
      (28, 1.202140956169)
      (29, 1.2722232054971)
      (30, 1.3486478783407)
      (31, 1.3727562816561)
      (32, 1.4078583915456)
      (33, 1.4271911802559)
      (34, 1.4583596377024)
      (35, 1.4943067887155)
      (36, 1.5156416288099)
      (37, 1.5673514418731)
      (38, 1.5257563653458)
      (39, 1.6285626741519)
      (40, 1.6015515358324)
      (41, 1.6398446846539)
      (42, 1.6105695090346)
      (43, 1.7357855482578)
      (44, 1.663257048135)
      (45, 1.7082183696376)
      (46, 1.7141818711466)
      (47, 1.7386976353862)
      (48, 1.7759606689769)
      (49, 1.7871046751784)
      (50, 1.8611995228113)
      (51, 1.8164003584035)
      (52, 1.8911369475636)
      (53, 1.9023625888079)
      (54, 1.9934069178642)
      (55, 1.9734976870415)
      (56, 2.0191634605762)
      (57, 1.9696427700865)
      (58, 2.0904006165687)
      (59, 2.0964187625424)
      (60, 2.1674069288648)
      (61, 2.1162466149076)
      (62, 2.188770066379)
      (63, 2.1755028772937)
    };
    %\addlegendentry{Self-adjusting}

  \end{groupplot}
\end{tikzpicture}

      \label{fig:classbench-fw1}
    }
  }
  \subfloat[][Synthetic traffic]{
    \resizebox{.225\textwidth}{!}{
      \pgfplotsset{
  RatePlot/.style = {
    tick pos = left,
    xtick align=outside,
    ytick align=outside,
    xlabel near ticks,
    ylabel near ticks,
    width=.4\textwidth,
    height=.3\textwidth,
    legend pos = north west,
    legend cell align=left,
    ylabel = {Throughput [Mpps]},
    xlabel = {number of CPU cores},
    xmin=1, xmax=32,
    ymin=0,
  },
  SpeedupPlot/.style = {
    RatePlot,
    ylabel={Speedup},
  },
  ClassBenchGroupPlot/.style = {
    group/group size = 1 by 2,
    group/horizontal sep = 0pt,
    group/vertical sep = 28pt,
  },
  ClassBenchRatePlot/.style = {
    RatePlot
  },
  ClassBenchSpeedupPlot/.style = {
    SpeedupPlot,
  }
}

%%% Local Variables:
%%% mode: latex
%%% TeX-master: "../distributed_mrf.tex"
%%% End:


\begin{tikzpicture}
  \begin{axis}[SpeedupPlot]

    \addplot coordinates {
     (1, 1.00000)
     (2, 3.95295)
     (3, 8.92334)
     (4, 14.46385)
     (5, 21.54861)
     (6, 31.60320)
     (7, 41.12957)
     (8, 50.19947)
     (9, 62.42447)
     (10, 75.20405)
     (11, 89.14996)
     (12, 104.37910)
     (13, 119.63990)
     (14, 135.54179)
     (15, 152.84439)
     (16, 170.62145)
     (17, 188.02705)
     (18, 206.09421)
     (19, 224.53568)
     (20, 242.09435)
     (21, 261.02735)
     (22, 283.27952)
     (23, 302.95757)
     (24, 323.22273)
     (25, 341.55632)
     (26, 358.29783)
     (27, 369.12554)
     (28, 376.03609)
     (29, 384.30787)
     (30, 394.90903)
     (31, 410.13636)
     (32, 421.21499)
     (33, 417.77358)
     (34, 412.84227)
     (35, 411.28094)
     (36, 416.44472)
     (37, 378.12284)
     (38, 409.36889)
     (39, 405.14475)
     (40, 405.59788)
     (41, 409.67940)
     (42, 419.91976)
     (43, 425.23958)
     (44, 423.57837)
     (45, 426.08588)
     (46, 422.50951)
     (47, 425.82855)
     (48, 424.13425)
     (49, 424.06522)
     (50, 425.31558)
     (51, 425.05500)
     (52, 423.83465)
     (53, 423.60081)
     (54, 419.90105)
     (55, 418.55585)
     (56, 413.35431)
     (57, 417.40235)
     (58, 417.75017)
     (59, 417.00940)
     (60, 416.22015)
     (61, 416.50215)
     (62, 418.20244)
     (63, 422.64641)
    };
    \addlegendentry{1997}

    \addplot coordinates {
     (1, 1.00000)
     (2, 4.05264)
     (3, 9.15030)
     (4, 15.96561)
     (5, 24.80474)
     (6, 34.34314)
     (7, 49.33390)
     (8, 58.57508)
     (9, 73.22804)
     (10, 89.05021)
     (11, 104.10408)
     (12, 123.56484)
     (13, 147.64577)
     (14, 181.78693)
     (15, 205.48899)
     (16, 224.80540)
     (17, 244.77251)
     (18, 257.70215)
     (19, 286.05802)
     (20, 310.46298)
     (21, 337.48591)
     (22, 363.81849)
     (23, 395.29418)
     (24, 424.94692)
     (25, 453.33684)
     (26, 485.48818)
     (27, 520.00243)
     (28, 553.12692)
     (29, 582.64861)
     (30, 614.52174)
     (31, 644.49851)
     (32, 678.03008)
     (33, 682.84231)
     (34, 700.69074)
     (35, 719.41674)
     (36, 731.14138)
     (37, 741.68328)
     (38, 756.32739)
     (39, 769.92112)
     (40, 794.13442)
     (41, 810.53042)
     (42, 824.67171)
     (43, 847.09128)
     (44, 835.86157)
     (45, 844.48204)
     (46, 846.74468)
     (47, 861.32880)
     (48, 859.15198)
     (49, 867.92826)
     (50, 860.93654)
     (51, 871.99385)
     (52, 874.04773)
     (53, 864.74101)
     (54, 878.46596)
     (55, 872.64876)
     (56, 884.06643)
     (57, 906.55279)
     (58, 920.26162)
     (59, 934.98393)
     (60, 884.32544)
     (61, 971.44811)
     (62, 991.51270)
     (63, 1005.94749)
   };
    \addlegendentry{4999}

    \addplot coordinates {
     (1, 1.00000)
     (2, 4.03001)
     (3, 9.11404)
     (4, 16.34207)
     (5, 25.21433)
     (6, 36.30401)
     (7, 48.99240)
     (8, 62.98751)
     (9, 79.45352)
     (10, 93.36512)
     (11, 112.40743)
     (12, 133.86550)
     (13, 161.87073)
     (14, 168.51244)
     (15, 198.94454)
     (16, 215.54795)
     (17, 240.27260)
     (18, 270.62833)
     (19, 306.97158)
     (20, 329.09305)
     (21, 384.44318)
     (22, 434.39370)
     (23, 482.61550)
     (24, 518.94320)
     (25, 564.19911)
     (26, 605.58748)
     (27, 639.31087)
     (28, 668.46241)
     (29, 674.14551)
     (30, 728.44121)
     (31, 755.55375)
     (32, 785.93100)
     (33, 790.11100)
     (34, 806.26242)
     (35, 828.58456)
     (36, 853.19989)
     (37, 861.55702)
     (38, 880.55055)
     (39, 901.53135)
     (40, 911.19900)
     (41, 936.73643)
     (42, 961.95934)
     (43, 973.47168)
     (44, 991.87322)
     (45, 1010.76263)
     (46, 1038.17350)
     (47, 1054.96610)
     (48, 1088.11901)
     (49, 1089.27339)
     (50, 1102.94533)
     (51, 1133.88167)
     (52, 1148.90173)
     (53, 1160.22556)
     (54, 1186.14237)
     (55, 1200.43832)
     (56, 1225.89162)
     (57, 1239.41472)
     (58, 1268.46348)
     (59, 1292.77520)
     (60, 1305.96162)
     (61, 1309.26166)
     (62, 1329.98301)
     (63, 1374.24497)
   };
    \addlegendentry{10007}

  \end{axis}
\end{tikzpicture}

      \label{fig:rule-size}
    }
  }
  \caption{ClassBench results on three rulesets with roughly 5000 rules coming from different seeds, and synthetic test with uniform traffic.}
  \label{fig:classbench}
\end{figure*}

\noindent
\textbf{Rule size.} %
\begin{itemize}
\item no dependency rule-set and uniform traffic
\item rule-template: 1.2.3.4+udp-dst:i[1,n], where n is a parameter: 997, 4999, 10007 (primes)
\item flows: uniform
\item RSS: 5-tuple hash
\item 1 fig with 3 plots for each rule size
\item takeaway: rule size does not matter
\end{itemize}

\noindent
\textbf{Active flow size.} %
\begin{itemize}
\item same no-dep rule-set (udp-dst) + 0, 5, 50, 500 flows per rule
\item RSS: 5-tuple hash
\item 1 fig with the 4 scaling laws, one for each active-flow-size: speedup (speedup: normalized for the single-core rate)
\item takeaway: the more flows the more rules are replicated across cores and the larger the working set size and superlinearity vanishes
\end{itemize}

\noindent%
\textbf{Rule dependencies.} %
\begin{itemize}
\item rule template: 1.2.3.4/{32,28,24,20,16,12,8,4,0}+udp-dst:i[1,500]
\item traffic template: uniform: 1.2.3.4:i (500 flows)
\item 4 rules-sets: no-dep, small-dep (/0 and /32 for each chain), medium-dep (/{32,24,16,8,0} for each chain), high-dep (/{32,28,24,20,16,12,8,4,0} for each chain)
\item RSS: 5-tuple hash
\item 1 fig with 4 scaling laws: speedup (normalized for the single-core rate)
\item takewaway: the more dependencies, the less superlinearity (dependencies must be in the working set at each thread) (but note that our high-dep rule sets have more rules!)
\end{itemize}

\noindent%
\textbf{Locality boosting.} % (1 fig)
\begin{itemize}
\item benchmark the high active flow-count benchmark with different RSS hashes: bad hash: IP-dst (everything goes to 1 cpu, no scaling), 5-tuple hash (imperfect partitioning in the LB), good hash: udp-dst hash (dep chains are partitioned across cpus: only a few chains per each CPU)
\item takeaway: locality boosting matters
\end{itemize}

\begin{figure*}
  \centering
  \subfloat[][Active flow size]{
    \resizebox{.225\textwidth}{!}{
      \pgfplotsset{
  RatePlot/.style = {
    tick pos = left,
    xtick align=outside,
    ytick align=outside,
    xlabel near ticks,
    ylabel near ticks,
    width=.4\textwidth,
    height=.3\textwidth,
    legend pos = north west,
    legend cell align=left,
    ylabel = {Throughput [Mpps]},
    xlabel = {number of CPU cores},
    xmin=1, xmax=32,
    ymin=0,
  },
  SpeedupPlot/.style = {
    RatePlot,
    ylabel={Speedup},
  },
  ClassBenchGroupPlot/.style = {
    group/group size = 1 by 2,
    group/horizontal sep = 0pt,
    group/vertical sep = 28pt,
  },
  ClassBenchRatePlot/.style = {
    RatePlot
  },
  ClassBenchSpeedupPlot/.style = {
    SpeedupPlot,
  }
}

%%% Local Variables:
%%% mode: latex
%%% TeX-master: "../distributed_mrf.tex"
%%% End:


\begin{tikzpicture}
  \begin{axis}[
    SpeedupPlot,
    legend cell align=right,
    ylabel=,
    ]

    \addplot[BenchMarkI] coordinates {
     (1, 1.00000)
     (2, 3.69271)
     (3, 7.59393)
     (4, 12.39640)
     (5, 17.85511)
     (6, 23.97603)
     (7, 31.00011)
     (8, 39.60983)
     (9, 47.61334)
     (10, 57.36113)
     (11, 68.72266)
     (12, 78.31285)
     (13, 92.15389)
     (14, 104.72159)
     (15, 118.39065)
     (16, 133.30634)
     (17, 147.96168)
     (18, 162.94917)
     (19, 178.62820)
     (20, 190.89709)
     (21, 202.64813)
     (22, 226.46461)
     (23, 244.11065)
     (24, 260.79442)
     (25, 276.92718)
     (26, 294.99096)
     (27, 315.54284)
     (28, 334.89944)
     (29, 351.44948)
     (30, 367.81864)
     (31, 382.18944)
     (32, 395.67767)
     (33, 396.44293)
     (34, 395.71645)
     (35, 395.67800)
     (36, 395.82455)
     (37, 400.67397)
     (38, 405.63883)
     (39, 408.93034)
     (40, 410.89831)
     (41, 414.06212)
     (42, 419.20309)
     (43, 427.84012)
     (44, 427.35800)
     (45, 424.89619)
     (46, 432.20417)
     (47, 422.73808)
     (48, 432.39426)
     (49, 430.94132)
     (50, 425.48526)
     (51, 421.98060)
     (52, 425.21264)
     (53, 428.98377)
     (54, 432.42837)
     (55, 434.75696)
     (56, 442.91349)
     (57, 447.24244)
     (58, 452.63150)
     (59, 456.84024)
     (60, 460.93679)
     (61, 465.87988)
     (62, 471.88347)
     (63, 475.99055)
    };
    \addlegendentry{1 flow/rule}

    \addplot[BenchMarkII] coordinates {
     (1, 1.00000)
     (2, 2.39525)
     (3, 3.71735)
     (4, 6.67169)
     (5, 7.61735)
     (6, 10.18570)
     (7, 13.69728)
     (8, 19.42190)
     (9, 21.01766)
     (10, 22.81520)
     (11, 25.69332)
     (12, 30.87965)
     (13, 32.88603)
     (14, 41.55270)
     (15, 43.36216)
     (16, 52.73299)
     (17, 53.42955)
     (18, 61.73582)
     (19, 60.58890)
     (20, 67.85519)
     (21, 70.76329)
     (22, 79.70996)
     (23, 87.19791)
     (24, 89.39942)
     (25, 100.48197)
     (26, 98.34627)
     (27, 103.84880)
     (28, 127.17754)
     (29, 115.84134)
     (30, 127.59229)
     (31, 125.03694)
     (32, 131.71328)
     (33, 133.87871)
     (34, 142.97915)
     (35, 141.81431)
     (36, 168.10608)
     (37, 148.69333)
     (38, 155.49255)
     (39, 156.16419)
     (40, 157.94666)
     (41, 163.02547)
     (42, 165.69095)
     (43, 165.35995)
     (44, 182.97805)
     (45, 175.15676)
     (46, 185.40087)
     (47, 182.80703)
     (48, 184.99366)
     (49, 190.40863)
     (50, 203.04476)
     (51, 199.95664)
     (52, 199.50576)
     (53, 206.22115)
     (54, 206.49067)
     (55, 209.51526)
     (56, 229.15803)
     (57, 212.32688)
     (58, 217.94033)
     (59, 222.64390)
     (60, 222.05691)
     (61, 226.50798)
     (62, 224.49734)
     (63, 225.63869)
    };
    \addlegendentry{5 flows/rule}

    \addplot[BenchMarkIII] coordinates {
     (1, 1.00000)
     (2, 1.99762)
     (3, 2.99696)
     (4, 4.09499)
     (5, 5.00791)
     (6, 6.04364)
     (7, 7.25373)
     (8, 8.43047)
     (9, 10.07052)
     (10, 10.31642)
     (11, 11.59775)
     (12, 13.49110)
     (13, 14.03755)
     (14, 16.73626)
     (15, 17.59756)
     (16, 20.99676)
     (17, 21.01936)
     (18, 24.06885)
     (19, 22.96698)
     (20, 25.10826)
     (21, 26.07306)
     (22, 27.91234)
     (23, 30.57591)
     (24, 32.42546)
     (25, 34.25522)
     (26, 33.51962)
     (27, 36.62771)
     (28, 37.21090)
     (29, 39.51543)
     (30, 41.61023)
     (31, 40.81449)
     (32, 46.01953)
     (33, 41.94063)
     (34, 45.34462)
     (35, 44.54715)
     (36, 45.57075)
     (37, 45.84628)
     (38, 46.66575)
     (39, 46.92773)
     (40, 46.84492)
     (41, 47.51799)
     (42, 47.56298)
     (43, 47.92702)
     (44, 49.35726)
     (45, 50.17035)
     (46, 51.09499)
     (47, 50.92344)
     (48, 50.83023)
     (49, 51.51420)
     (50, 54.08494)
     (51, 52.38056)
     (52, 53.05347)
     (53, 55.45460)
     (54, 53.26248)
     (55, 56.10367)
     (56, 59.17707)
     (57, 57.72334)
     (58, 55.94015)
     (59, 57.40285)
     (60, 57.54928)
     (61, 57.63440)
     (62, 58.68111)
     (63, 59.12448)
   };
    \addlegendentry{50 flows/rule}

    \addplot[BenchMarkIV] coordinates {
     (1, 1.00000)
     (2, 1.99992)
     (3, 3.00011)
     (4, 4.07772)
     (5, 4.98913)
     (6, 5.97250)
     (7, 6.96411)
     (8, 8.14096)
     (9, 9.11581)
     (10, 9.90044)
     (11, 10.91903)
     (12, 12.13378)
     (13, 12.89141)
     (14, 13.98891)
     (15, 15.23297)
     (16, 16.88147)
     (17, 17.21213)
     (18, 18.42680)
     (19, 18.97400)
     (20, 20.20907)
     (21, 20.95819)
     (22, 21.91435)
     (23, 23.29689)
     (24, 24.58443)
     (25, 25.47443)
     (26, 26.16301)
     (27, 27.18897)
     (28, 28.71772)
     (29, 29.03039)
     (30, 29.78861)
     (31, 30.50235)
     (32, 33.82553)
     (33, 31.29497)
     (34, 31.88813)
     (35, 31.78408)
     (36, 33.02912)
     (37, 32.03299)
     (38, 31.66617)
     (39, 32.34055)
     (40, 32.39495)
     (41, 32.74973)
     (42, 32.25115)
     (43, 32.84170)
     (44, 33.63734)
     (45, 33.22351)
     (46, 33.83922)
     (47, 33.68107)
     (48, 34.47694)
     (49, 33.98644)
     (50, 34.33626)
     (51, 34.32101)
     (52, 34.64329)
     (53, 35.05040)
     (54, 34.46578)
     (55, 35.33397)
     (56, 35.26599)
     (57, 35.64467)
     (58, 34.98498)
     (59, 35.76840)
     (60, 36.51876)
     (61, 35.98503)
     (62, 36.52747)
     (63, 36.85263)
   };
    \addlegendentry{500 flows/rule}

  \end{axis}
\end{tikzpicture}

      \label{fig:active-flow-size}
    }
  }
  \subfloat[][Rule dependency]{
    \resizebox{.225\textwidth}{!}{
      \pgfplotsset{
  RatePlot/.style = {
    tick pos = left,
    xtick align=outside,
    ytick align=outside,
    xlabel near ticks,
    ylabel near ticks,
    width=.4\textwidth,
    height=.3\textwidth,
    legend pos = north west,
    legend cell align=left,
    ylabel = {Throughput [Mpps]},
    xlabel = {number of CPU cores},
    xmin=1, xmax=32,
    ymin=0,
  },
  SpeedupPlot/.style = {
    RatePlot,
    ylabel={Speedup},
  },
  ClassBenchGroupPlot/.style = {
    group/group size = 1 by 2,
    group/horizontal sep = 0pt,
    group/vertical sep = 28pt,
  },
  ClassBenchRatePlot/.style = {
    RatePlot
  },
  ClassBenchSpeedupPlot/.style = {
    SpeedupPlot,
  }
}

%%% Local Variables:
%%% mode: latex
%%% TeX-master: "../distributed_mrf.tex"
%%% End:


\begin{tikzpicture}
  \begin{axis}[
    SpeedupPlot,
    legend cell align = left,
    ]

   \addplot coordinates {
     (1, 1.00000)
     (2, 4.13055)
     (3, 9.21117)
     (4, 16.30970)
     (5, 24.81983)
     (6, 35.45684)
     (7, 50.91347)
     (8, 59.75825)
     (9, 76.35076)
     (10, 93.30943)
     (11, 110.20833)
     (12, 134.18967)
     (13, 166.09775)
     (14, 188.66353)
     (15, 202.32700)
     (16, 227.97109)
     (17, 251.23778)
     (18, 272.36801)
     (19, 303.37045)
     (20, 331.29609)
     (21, 359.75187)
     (22, 388.76623)
     (23, 421.06514)
     (24, 455.53010)
     (25, 488.13324)
     (26, 519.68040)
     (27, 550.32538)
     (28, 591.71689)
     (29, 624.99445)
     (30, 656.72324)
     (31, 689.77010)
     (32, 720.69800)
     (33, 733.83137)
     (34, 750.03616)
     (35, 764.48335)
     (36, 784.97749)
     (37, 788.74272)
     (38, 812.70432)
     (39, 828.42427)
     (40, 839.94883)
     (41, 850.44871)
     (42, 868.26194)
     (43, 878.55392)
     (44, 885.03494)
     (45, 874.10334)
     (46, 876.38467)
     (47, 876.43419)
     (48, 891.31902)
     (49, 887.92400)
     (50, 887.23406)
     (51, 871.87189)
     (52, 874.97299)
     (53, 884.71108)
     (54, 849.13485)
     (55, 911.41042)
     (56, 926.55894)
     (57, 945.90496)
     (58, 962.79104)
     (59, 983.07913)
     (60, 999.62159)
     (61, 1015.55166)
     (62, 1032.86080)
     (63, 1049.23158)
   };
   \addlegendentry{1 dep}

   \addplot coordinates {
     (1, 1.00000)
     (2, 4.20956)
     (3, 9.80122)
     (4, 18.02564)
     (5, 28.02049)
     (6, 40.21398)
     (7, 54.16830)
     (8, 70.30137)
     (9, 91.46462)
     (10, 113.73782)
     (11, 128.94203)
     (12, 150.10982)
     (13, 175.34445)
     (14, 201.94204)
     (15, 226.65010)
     (16, 257.95223)
     (17, 306.56251)
     (18, 354.73212)
     (19, 387.34385)
     (20, 418.16182)
     (21, 448.06770)
     (22, 481.46649)
     (23, 517.33749)
     (24, 546.63846)
     (25, 588.55278)
     (26, 627.86035)
     (27, 670.74695)
     (28, 711.78719)
     (29, 751.93859)
     (30, 789.60831)
     (31, 828.79598)
     (32, 867.60502)
     (33, 887.21664)
     (34, 909.83037)
     (35, 944.26035)
     (36, 964.76735)
     (37, 992.50226)
     (38, 1024.84776)
     (39, 1044.49203)
     (40, 1069.06362)
     (41, 1092.10196)
     (42, 1124.16993)
     (43, 1122.83936)
     (44, 1160.12321)
     (45, 1177.58792)
     (46, 1196.32096)
     (47, 1213.39914)
     (48, 1233.29483)
     (49, 1244.95859)
     (50, 1274.65455)
     (51, 1288.31785)
     (52, 1310.81708)
     (53, 1297.36010)
     (54, 1310.64612)
     (55, 1335.55162)
     (56, 1361.17607)
     (57, 1396.46381)
     (58, 1428.29590)
     (59, 1372.08098)
     (60, 1476.04703)
     (61, 1502.88471)
     (62, 1515.61569)
     (63, 1538.06538)
   };
   \addlegendentry{2 deps}

   \addplot coordinates {
     (1, 1.00000)
     (2, 4.09559)
     (3, 9.74382)
     (4, 18.06109)
     (5, 28.82982)
     (6, 42.28299)
     (7, 58.62579)
     (8, 77.08843)
     (9, 99.46497)
     (10, 121.35392)
     (11, 143.65033)
     (12, 171.33224)
     (13, 207.80672)
     (14, 231.11656)
     (15, 260.76345)
     (16, 297.31088)
     (17, 348.14032)
     (18, 400.84988)
     (19, 440.72388)
     (20, 475.93486)
     (21, 502.30099)
     (22, 541.08979)
     (23, 600.93983)
     (24, 645.57302)
     (25, 695.91155)
     (26, 747.88493)
     (27, 814.50663)
     (28, 878.21680)
     (29, 937.24575)
     (30, 991.81551)
     (31, 1034.79970)
     (32, 1096.02551)
     (33, 1126.72588)
     (34, 1157.68793)
     (35, 1194.52271)
     (36, 1233.64986)
     (37, 1280.82266)
     (38, 1316.85801)
     (39, 1351.11137)
     (40, 1369.66417)
     (41, 1425.06949)
     (42, 1443.31706)
     (43, 1480.45583)
     (44, 1528.72646)
     (45, 1554.30917)
     (46, 1608.87202)
     (47, 1646.79727)
     (48, 1676.55238)
     (49, 1726.39835)
     (50, 1739.21886)
     (51, 1813.05113)
     (52, 1854.59844)
     (53, 1890.66197)
     (54, 1933.87081)
     (55, 1977.37758)
     (56, 1991.26416)
     (57, 2054.33494)
     (58, 2095.38732)
     (59, 2077.32428)
     (60, 2162.15475)
     (61, 2196.91912)
     (62, 2233.75986)
     (63, 2266.89573)
    };
    \addlegendentry{4 deps}

    \addplot coordinates {
      (1, 1.00000)
      (2, 4.03643)
      (3, 9.42434)
      (4, 17.29031)
      (5, 27.42397)
      (6, 40.01640)
      (7, 55.42011)
      (8, 73.39404)
      (9, 92.41235)
      (10, 116.04810)
      (11, 141.43186)
      (12, 170.55191)
      (13, 198.45991)
      (14, 233.52928)
      (15, 266.65325)
      (16, 300.39072)
      (17, 339.46017)
      (18, 376.49776)
      (19, 428.68548)
      (20, 473.66299)
      (21, 527.09097)
      (22, 574.12143)
      (23, 631.80781)
      (24, 681.29106)
      (25, 725.55390)
      (26, 783.98860)
      (27, 860.84777)
      (28, 921.18695)
      (29, 1008.30603)
      (30, 1063.39555)
      (31, 1128.00671)
      (32, 1175.51284)
      (33, 1204.60860)
      (34, 1263.83283)
      (35, 1281.56024)
      (36, 1314.20549)
      (37, 1342.18224)
      (38, 1408.76478)
      (39, 1441.80773)
      (40, 1482.83015)
      (41, 1514.18160)
      (42, 1546.11455)
      (43, 1592.95366)
      (44, 1643.98006)
      (45, 1701.52974)
      (46, 1751.03223)
      (47, 1805.40943)
      (48, 1875.47780)
      (49, 1915.15688)
      (50, 1941.78121)
      (51, 1990.36893)
      (52, 2061.46297)
      (53, 2105.21388)
      (54, 2160.57518)
      (55, 2197.68336)
      (56, 2223.15519)
      (57, 2282.98766)
      (58, 2330.94532)
      (59, 2366.03522)
      (60, 2409.43767)
      (61, 2448.62899)
      (62, 2480.89889)
      (63, 2522.01054)
    };
    \addlegendentry{8 deps}

  \end{axis}
\end{tikzpicture}

      \label{fig:rule-dependencies}
    }
  }
  \subfloat[][Zipf traffic (1997 rules)]{
    \resizebox{.225\textwidth}{!}{
      \pgfplotsset{
  RatePlot/.style = {
    tick pos = left,
    xtick align=outside,
    ytick align=outside,
    xlabel near ticks,
    ylabel near ticks,
    width=.4\textwidth,
    height=.3\textwidth,
    legend pos = north west,
    legend cell align=left,
    ylabel = {Throughput [Mpps]},
    xlabel = {number of CPU cores},
    xmin=1, xmax=32,
    ymin=0,
  },
  SpeedupPlot/.style = {
    RatePlot,
    ylabel={Speedup},
  },
  ClassBenchGroupPlot/.style = {
    group/group size = 1 by 2,
    group/horizontal sep = 0pt,
    group/vertical sep = 28pt,
  },
  ClassBenchRatePlot/.style = {
    RatePlot
  },
  ClassBenchSpeedupPlot/.style = {
    SpeedupPlot,
  }
}

%%% Local Variables:
%%% mode: latex
%%% TeX-master: "../distributed_mrf.tex"
%%% End:


\begin{tikzpicture}
  \begin{axis}[
    SpeedupPlot,
    legend cell align = right,
   ]

    \addplot coordinates {
    };
    \addlegendentry{Static}

    \addplot coordinates {
    };
    \addlegendentry{Self-adjusting}

  \end{axis}
\end{tikzpicture}

      \label{fig:zipf}
    }
  }
  \subfloat[][Locality boosting (RSS)]{
    \resizebox{.225\textwidth}{!}{
      \pgfplotsset{
  RatePlot/.style = {
    tick pos = left,
    xtick align=outside,
    ytick align=outside,
    xlabel near ticks,
    ylabel near ticks,
    width=.4\textwidth,
    height=.3\textwidth,
    legend pos = north west,
    legend cell align=left,
    ylabel = {Throughput [Mpps]},
    xlabel = {number of CPU cores},
    xmin=1, xmax=32,
    ymin=0,
  },
  SpeedupPlot/.style = {
    RatePlot,
    ylabel={Speedup},
  },
  ClassBenchGroupPlot/.style = {
    group/group size = 1 by 2,
    group/horizontal sep = 0pt,
    group/vertical sep = 28pt,
  },
  ClassBenchRatePlot/.style = {
    RatePlot
  },
  ClassBenchSpeedupPlot/.style = {
    SpeedupPlot,
  }
}

%%% Local Variables:
%%% mode: latex
%%% TeX-master: "../distributed_mrf.tex"
%%% End:


\begin{tikzpicture}
  \begin{axis}[
    SpeedupPlot,
    legend cell align = right,
    ylabel=,
    %xmax=23,
   ]

   \addplot[BenchMarkI] coordinates {
     (1, 1.00000)
     (2, 3.95142)
     (3, 8.68203)
     (4, 14.90004)
     (5, 22.47613)
     (6, 31.38553)
     (7, 42.17986)
     (8, 51.72500)
     (9, 66.01503)
     (10, 79.53602)
     (11, 95.35325)
     (12, 111.60849)
     (13, 128.37300)
     (14, 145.75764)
     (15, 133.72694)
     (16, 175.45037)
     (17, 149.00504)
     (18, 202.86551)
     (19, 205.22914)
     (20, 243.57678)
     (21, 252.52939)
     (22, 223.29265)
     (23, 252.09666)
     (24, 306.34975)
     (25, 271.85857)
     (26, 214.73210)
     (27, 317.23979)
     (28, 315.68316)
     (29, 305.06051)
     (30, 275.47492)
     (31, 347.29775)
     (32, 400.34266)
     (33, 356.08968)
     (34, 302.47047)
     (35, 306.59236)
     (36, 352.21984)
     (37, 339.88198)
     (38, 333.94146)
     (39, 355.68272)
     (40, 377.99873)
     (41, 361.28744)
     (42, 351.05213)
     (43, 340.82214)
     (44, 315.81173)
     (45, 345.04018)
     (46, 359.93693)
     (47, 374.58809)
     (48, 373.26345)
     (49, 368.11506)
     (50, 371.11681)
     (51, 350.22940)
     (52, 317.03837)
     (53, 351.15369)
     (54, 379.84346)
     (55, 370.15382)
     (56, 363.39852)
     (57, 371.21767)
     (58, 372.44045)
     (59, 363.96681)
     (60, 372.05666)
     (61, 365.32096)
     (62, 371.30476)
     (63, 362.81837)
    };
    \addlegendentry{dstUDP hash}

    \addplot[BenchMarkII] coordinates {
           (1, 1.00000)
     (2, 1.98130)
     (3, 3.00769)
     (4, 3.97561)
     (5, 5.00431)
     (6, 5.99609)
     (7, 6.97018)
     (8, 7.92406)
     (9, 8.95830)
     (10, 9.91184)
     (11, 10.75034)
     (12, 11.84526)
     (13, 12.78534)
     (14, 13.75559)
     (15, 14.79208)
     (16, 15.70733)
     (17, 16.69676)
     (18, 17.70482)
     (19, 18.67260)
     (20, 19.61432)
     (21, 20.35937)
     (22, 21.53298)
     (23, 22.54723)
     (24, 23.39779)
     (25, 24.40678)
     (26, 25.36017)
     (27, 26.18343)
     (28, 26.86551)
     (29, 27.70759)
     (30, 28.16335)
     (31, 29.24560)
     (32, 30.18676)
     (33, 30.19869)
     (34, 30.35795)
     (35, 30.30901)
     (36, 30.49475)
     (37, 30.59769)
     (38, 30.72566)
     (39, 30.08395)
     (40, 30.80557)
     (41, 30.87025)
     (42, 30.68596)
     (43, 31.00707)
     (44, 31.12594)
     (45, 30.74867)
     (46, 31.53930)
     (47, 31.41372)
     (48, 31.55827)
     (49, 31.18007)
     (50, 31.70955)
     (51, 31.86570)
     (52, 31.99009)
     (53, 32.01835)
     (54, 32.10412)
     (55, 32.41636)
     (56, 32.56264)
     (57, 32.01769)
     (58, 32.30162)
     (59, 32.86291)
     (60, 33.15574)
     (61, 33.35238)
     (62, 32.90887)
     (63, 33.29944)
   };
   \addlegendentry{5-tuple hash}

  \end{axis}
\end{tikzpicture}

      \label{fig:locality-boosting}
    }
  }
  \caption{Microbenchmark results.}
  \label{fig:microbenchmark}
\end{figure*}


%%% Local Variables:
%%% mode: latex
%%% TeX-master: "distributed_mrf"
%%% End:
