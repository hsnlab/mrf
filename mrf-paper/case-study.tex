\section{Case-study }\label{sec:case-study}


self-adjusting algs and applications:
- linear rule matching: IDS, expert systems, rule-based/explainable AI inference
- IP FIBs: splay trees
- packet classification:

% Journal of Information Assurance and Security 4 (2009) 133-141
% Received June 10, 2009 1554-1010 $ 03.50 Dynamic Publishers, Inc
% Dynamic Scheme for Packet Classification Using
% Splay Trees

% In network applications, however, where static linked lists enjoy wide applicability, e.g., for rule matching in OpenFlow and P4 reference software switches, packet classification in the Linux OS network stack (\texttt{iptables}), etc., so far we haven't seen many uses of the self-adjusting version, i.e., MTF lists. We imagine potential applications in packet classification (see later in \S\ref{sec:packet-classifier}), flow table lookup, evaluating rules in an intrusion detection system, etc.; 


\subsection{Self-adjusting packet classifiers}
\label{sec:sa-nf-tables}

\subsection{Implementation}
\label{sec:sa-nf-tables-impl}

\subsection{Evaluation}
\label{sec:sa-nf-tables-eval}

- synthetic microbenchmarks and realistic macrobenchmark: rulesets and traffic

% when superlinear scaling appears
\noindent%
\textbf{Superlinear scaling.} %  (3 figs + 1 with Jonas's stats?)
\begin{itemize}
\item no dependency rule-set and uniform traffic
\item rule-template: 1.2.3.4+udp-dst:i[1,n], where n is a parameter: 997, 4999, 10007 (primes)
\item flows: uniform + zipf: 20000 * packets=pcap
\item RSS: 5-tuple hash
\item one fig per rule size (packet rate): 4 plots: SA+uniform, SA+zipf, baseline+uniform,baseline+zipf
\item takeaway: rule size + traffic-locality do not matter (3 figs + 1 with Jonas's stats?)
\end{itemize}

% when superlinear scaling deteriorates into linear because SA does not work
\noindent%
\textbf{Active flow size.} %  (1 fig)
\begin{itemize}
\item same no-dep rule-set (udp-dst) + 5, 50, 500 flows per rule
\item RSS: 5-tuple hash
\item fig: 3 scaling laws, one for each active-flow-size: speedup (speedup: normalized for the single-core rate)
\item takeaway: the more the flow count the more rules are replicated across cores and the larger the working set size and superlinearity vanishes
\end{itemize}

\noindent%
\textbf{Rule dependencies.} % (1 fig)
\begin{itemize}
\item rule template: 1.2.3.4/{32,28,24,20,16,12,8,4,0}+udp-dst:i[1,500]
\item   traffic template: uniform: 1.2.3.4:i (500 flows) 
\item   3 rules-sets: small-dep (/0 and /32 for each chain), medium-dep (/{32,24,16,8,0}  for each chain), high-dep (/{32,28,24,20,16,12,8,4,0}  for each chain)
\item   RSS: 5-tuple hash
\item: fig: 3 scaling laws: speedup (normalized for the single-core rate)
\item takewaway: the more dependencies, the less superlinearity (dependencies must be in the working set at each thread)
\end{itemize}

\noindent%
\textbf{Locality boosting.} % (1 fig)
\begin{itemize}
\item benchmark the high-dep rule-set with different RSS hashes: bad hash: IP-dst(everything goes to 1 cpu, no scaling), 5-tuple hash (same as in the previous case), good hash: udp-dst hash (dep chains are partitioned across cpus: only a few chains per each CPU) 
\item  takeaway: load-balancing policy matters, the less locality-boosting, the less superlinearity
\end{itemize}

% realistic macrobenchmark
\noindent%
\textbf{Real workload.} % (3 figs)
classbench: the good news (3 figs with 3 different seeds + 5000 rules) + throughput and latency diagrams

%%% Local Variables:
%%% mode: latex
%%% TeX-master: "distributed_mrf"
%%% End:

