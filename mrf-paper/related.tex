\section{Related work}
\label{sec:related-work}

superlinear scaling:

% general argument:
%
% Fixed Time, Tiered Memory, and Superlinear Speedup
% John L. Gustafson
% Ames Laboratory-USDOE
% Ames, IA 50011
% http://www.johngustafson.net/pubs/pub26/Superlinear.pdf

% Modeling Speedup (n) Greater than n
% IEEE Transactions on Parallel and Distributed Systems · May 1990
% DOI: 10.1109/71.80148 · Source: IEEE Xplore
% David P. Helmbold
% Charles E. Mcdowell
% https://www.researchgate.net/profile/David-Helmbold/publication/3299493_Modeling_Speedup_n_Greater_than_n/links/5509aca30cf26198a6391fcb/Modeling-Speedup-n-Greater-than-n.pdf

% systematic analysis, focusing on the cloud and data analytics
%
% Superlinear Speedup in HPC Systems: why and when?
% Sasko Ristov, Radu Prodan, Marjan Gusev, Karolj Skala
% https://annals-csis.org/Volume_8/pliks/498.pdf

% M. Gusev and S. Ristov, “Superlinear speedup in Windows Azure cloud,”
% in Cloud Networking (IEEE CLOUDNET), 2012 IEEE 1st International
% Conference on, Paris, France, 2012, pp. 173–175.
% https://www.researchgate.net/publication/258770134_Superlinear_Speedup_in_Windows_Azure_Cloud

% NEIL J. GUNTHER, PAUL PUGLIA, AND KRISTOFER TOMASETTE
% Hadoop Superlinear Scalability
% ACQ Queue
% https://dl.acm.org/doi/pdf/10.1145/2719919

% for caching
%
% The Elements of Cache Programming Style
% Chris B. Sears
% Google Inc.
% https://www.usenix.org/legacy/publications/library/proceedings/als00/2000papers/papers/full_papers/sears/sears.pdf

% in robot control
%
% http://heikohamann.de/pub/hamannARCS2018.pdf
% Superlinear Scalability in Parallel Computing and Multi-Robot Systems: Shared Resources, Collaboration, and Network Topology
% Heiko Hamann
% http://heikohamann.de/pub/hamannARCS2018.pdf

% impossible:
% Faber, V., Lubeck, O. M., and White, A. B., “Superlinear Speedup of an Efficient Sequential
% Algorithm is Not Possible,” Parallel Computing, 3 (1986), pp. 259–260.

% no application in networking

caching in networking:
\begin{itemize}
\item state caching in NFs: https://www.usenix.org/conference/nsdi24/presentation/lei
\item SQL caches: redis, memcached
\item distributed web caching and CDNs: https://www.usenix.org/conference/nsdi24/presentation/zhang-yazhuo
\item dataplane kv store caching (NSDI)
\item caching in microservices: https://www.usenix.org/conference/nsdi24/presentation/zhang-haoran
\end{itemize}

load-balancing:
\begin{itemize}
\item Justin's ccr paper
\item RSS++
\end{itemize}

sa classifiers:
\begin{itemize}
\item our INFOCOM paper
\item any other SA paper from the systems literature?
\end{itemize}

% \subsection{Self-adjusting Data Structures}

% Self-adjusting data structures are widely used in algorithms and computer systems.
% Even the simplest data structure, self-adjusting is used in e.g. (1) a subroutine in computing point maxima and convex hulls~\cite{BentleyCL93}, (2) organizing lists of identifiers in program compilation and interpretation~\cite{HesterH85} or (3) detecting collisions in hash tables~\cite{HesterH85}.
% Some compression schemes use self-adjusting lists~\cite{BentleySTW86}.
% The problem of managing cache~\cite{SleatorT85} can be viewed as a self-adjusting data structure (every algorithm for list reorganization gives rise to a cache management algorithm).
% Further self-adjusting data structures include splay trees~\cite{SleatorT85Splay}, self-adjusting skip lists~\cite{BoseDL08}, push-down trees~\cite{Avin0020}, or self-adjusting trees for storing geometric data~\cite{ParkM12}.


% Models of self-adjusting data structures are based on the cost of access and rearrangement. For example:

% \begin{enumerate}
%   \item \textbf{Self-adjusting lists}~\cite{SleatorT85}. We are given a set of items, arranged in a linear list, and a sequence of access requests $\sigma$ to the nodes of the list.
% Upon receiving an access request to a node in
% the list, an algorithm searches linearly through the list, starting
% from the head of the list, traversing nodes until encountering the
% accessed node. Accessing the node at position i in the list costs i
% (the first node is at position 1).
% After serving a request, an algorithm may
% choose to rearrange the nodes of the list, paying the cost 1 per each transposition of neighboring items. 


% \item \textbf{Binary search trees}~\cite{SleatorT85Splay}.
% When the universe of items is ordered, we may store them in a binary search tree.
% A classic binary search tree is a \emph{splay tree}~\cite{SleatorT85Splay}.
% Another important search tree is $O(\log \log n)$-competitive \emph{tango tree}~\cite{demaine2007dyynamic}.
% The dynamic optimality conjecture~\cite{SleatorT85Splay} (does an $O(1)$-competitive algorithm exist?) is a major unresolved question, in contrast to the simpler self-adjusting lists setting.
% Splay trees have other properties, e.g. working set bounds, static optimality~\cite{SleatorT85Splay} and other.



% \item \textbf{Other self-adjusting data structures}. 
% Self-adjusting skip lists~\cite{BoseDL08} have an equivalent of the working set bound of splay trees.
% Push-down trees~\cite{Avin0020} are dynamically optimal and have the working set bound.
% Adaptive geometric space partitioning data structures exist, e.g. self-adjusting trees for storing geometric data~\cite{ParkM12}.
% The online metrical task system model~\cite{Borodin1992} underpins all these models, and captures generalizations such as caching, which has self-adjusting algorithms such as LRU~\cite{SleatorT85}.
% \end{enumerate}


% % Other examples: intrusion detection as mtflist, flow table lookup as splay tree, etc.
% % Each have their own challenges, and our model is just an example.

% \paragraph*{Locality.}
% Common inputs have high locality, i.e. the same items are accessed repeatedly.
% The locality parameter of input is often the determining factor for the performance of self-adjusting data structures (e.g. there exist arguments of locality for self-adjusting lists~\cite{AlbersL16}, working set bounds for splay trees~\cite{SleatorT85Splay} and paging~\cite{AlbersFG05}).


% \subsection{Load Balancing and Scaling}

% \paragraph*{Load balancing with random hash functions.}
% A random load balancing assignment function is sufficient to load-balance correctly.
% Gonnet~\cite{Gonnet81} proved that when throwing $n$ balls uniformly and independently at random into $n$ bins, the fullest bin has
% load $(1 + o(1)) \log n/ \log \log n$ in expectation.
% The maximum bin load with this approach is $O(\log n/ \log \log n)$ with high probability~\cite{DubhashiR98}.

% \paragraph*{Practical load balancing.}
% RSS+ paper~\cite{BarbetteKMK19}.

% \subsection{Packet classification}

% Various data structures for packet classification were proposed in the literature: lists, tries, hash tables, bit vectors, or decision trees~\cite{gupta2001algorithms,Srinivasan1999,Eppstein2001}, as well as hardware solutions (TCAM).
% Packet classifiers are often accompanied by caching systems that provide some adjustability to traffic.
% Due to its simplicity, a~linear lookup structure is commonly applied in practice, e.g., in the default firewall suite of the Linux operating system kernel called \texttt{iptables}~\cite{MianoBRBLP19}, the OpenFlow reference switch~\cite{openflow}, and in many QoS classifiers.

% retrogade speedup and the USL

% superlinear scaling in social sciences, "scaling human communities in large cities"


%%% Local Variables:
%%% mode: latex
%%% TeX-master: "distributed_mrf"
%%% End:

