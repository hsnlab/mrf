% !TEX ROOT = ./distributed_mrf.tex
\section{Conclusions}\label{sec:conclusions}

In this paper, we provide theoretical and empirical evidence that combining locality-boosting load balancing with parallel self-adjusting algorithms can yield faster-than-linear speedup in certain applications and under specific workload conditions. Our main contribution is the identification of the key design pattern that consistently appears in use cases where superlinear scaling emerges, along with a characterization of the conditions required for it to occur. Through extensive simulations, we demonstrate that this optimization technique can achieve scaling significantly beyond what has previously been observed, and we further illustrate its applicability on several widely used production applications. We extend the default \nftables Linux subsystem into a true self-adjusting packet classifier, which we use to identify the main workload characteristics (rule-dependency, flow diversity) that affect superlinear growth trends.  We also reproduce faster-than-linear scaling on a Memcached+PostgreSQL distributed storage system. Future research will be needed to apply our methodology in a broader range of use cases: for instance, rule-based network intrusion detection systems like Snort or Suricata \cite{10.5555/2537857.2537883} or explainable AI inferencing seem like appealing application candidates. In addition, identifying other reusable design patterns that enable superlinear scaling would be a valuable direction.

%%% Local Variables:
%%% mode: latex
%%% TeX-master: "distributed_mrf.nsdi"
%%% End:

