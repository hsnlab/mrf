\begin{small}
  \tikzmath
  {
    function lookup(\x)
    {
      if (\x < 10) then
      {
        return 0.1+0.9*(0.1*\x +(1-0.1*\x)*10)/\x;
      } else {
        return 0.1 + 0.9/\x;
      };
    };
    function dcache(\x)
    {
      return lookup(1.0)/lookup(\x);
    };
    function rrcache(\x)
    {
      return lookup(1.0)/(0.1+0.9*(0.1 +(1-0.1)*10)/\x);
    };
    function pcache(\x)
    {
      return lookup(1.0)/(0.1+0.9/\x);
    };
    \a = dcache(1);
    \b = lookup(1.0);
  }
  \begin{tikzpicture}
    \begin{axis}[
      width=250pt,
      height=170pt,
      xlabel={\#threads},
      ylabel={Speedup},
      xlabel near ticks,
      ylabel near ticks,
      xmin=0,
      xmax=20,
      ymin=0,
      ymax=67,
      xtick={1,5,10,15,20},
      legend style = {
        anchor = north west,
        at = {(0.01, 1.01)},
        font=\scriptsize,
        % draw = none,
      },
      % no markers
      ]
      \addplot[domain=0:25,black,solid]{dcache(x)};
      \addlegendentry{Hash-based load balancing}
      \addplot[domain=0:25,black,densely dotted]{rrcache(x)};
      \addlegendentry{Random load balancing}
      \addplot[domain=0:25,black,densely dashed]{pcache(x)};
      \addlegendentry{All requests hit the cache}
      % \node at (25,25) {\a, \b};
    \end{axis}
  \end{tikzpicture}
\end{small}

%%% Local Variables:
%%% mode: latex
%%% TeX-master: "../distributed_mrf"
%%% End:
