\appendix

% todo: add references. Also Albers locality for caching and list access for exact characterization

% todo: draw a graph
% todo: redraw the locality boosting with domains and minimal notation?

% todo: write that Splay trees may escape this theorem, and perform even better: its working set property is only an upper bound.

% todo: write about submodularity of the cost function

% todo: packet classification is not like list, remove from examples below

% todo: write in terms of universes, becasue the main body uses this language. Generalize from there

% todo: write that the load balancer partitions the universes

\section{Superlinear Scaling of Self-adjusting\\ Distributed Systems}
\label{sec:arch-scaling}


Scaling is defined as the ratio of the completion time of the baseline system with a single machine to the completion time of the system with $k$ machines, $S(k) = T(1) / T(k)$.
Consider a single job $j$ consisting of the \emph{serial} part $s$ and the \emph{parallel} part $1-s$.
Assuming fixed problem size, task independence and divisible work, Amdahl's law states that the completion time of the system with $k$ machines cannot scale linearly
\begin{equation*}\label{eq:mtf-perf}
  \frac{T(1)}{T(k)} \le \frac1{s + \frac{1-s}{k}} \enspace .
\end{equation*}
Superlinear scaling of the parallel part of the workload would occur if $S(k) = \frac1{s + \frac{1-s}{g(k)}}$ for some $g(k) > k$. Amdahl's law rules this out.


Self-adjusting architectures seem to escape Amdahl's law, and we observe superlinear scaling in our experiments.
We observe empirical superlinear scaling for distributed caching systems~\cite{271208, 10.5555/1012889.1012894, dobb-2}. To learn more, we empirically examine an example, \emph{distributed self-adjusting list lookup}, along with a performance analysis as demonstration. Our architecture consists of a locality-boosting partitioning load-balancer (see Fig.~\ref{fig:locality-boosting-lb}) combined with a self-adjusting move-to-front list (see Fig.~\ref{fig:mtf-example}) implemented in the workers.
Similarly to distributed caching, we find that the list lookup may scale its workload superlinearly with $k$ for some inputs.

This evidence suggests to state a hypothesis that the system may achieve superlinear scaling, e.g. quadratic:
\begin{equation*}\label{eq:mtf-perf}
  \frac{T(1)}{T(k)} \overset{?}{\ge} \frac1{s + \frac{1-s}{k^2}} \enspace .
\end{equation*}
Is this possible? We investigate this question in the following sections.
We obtain a negative and a positive result:
\begin{enumerate}
\item Superlinear scaling turns out to be impossible.
Our analysis  stands that this superlinear scaling cannot continue for arbitrarily large $k$.
Rather, the superlinear scaling is just an initial phenomenon: scaling initially grows superlinearly, but eventually hits a limit for some value of $k$. See the details in Theorem~\ref{thm:superlinear}.

\item
Nonetheless, we can reap the fruits of superlinear scaling within the initial values of $k$. When load balancers boost locality of the input stream, performance of distributed self-adjusting data structures improve, and we observe superlinear scaling. For parallel self-adjusting lists, improvements can amount up to $O(k^2)$ (see Observation~\ref{obs:list-k2}).
\end{enumerate}
To rigorously show these claims, we formulate the following mathematical model.


% The rationale for why this design achieves superlinear scaling is the following.
% Then, superlinear speedup is merely a product of two simultaneous $k\times$ speedup factors: one $k\times$ speedup comes from the self-adjusting list getting progressively faster as we add new workers (recall the ``scaled size'' model from \S\ref{sec:backgound-dist-cache}), and another $k\times$ speedup because we extend the total compute capacity available to the system $k$ times. The effective speedup is then just the multiple of the two, yielding $k^2$ times speedup in total. Plugging into Amdahl's law we get the \emph{scaling law for distributed MTF lists on uniform input} (see Fig.~\ref{fig:amdahl}):
% \begin{equation}\label{eq:mtf-perf}
  % S_l(k) = \frac{T_l(1)}{T_l(k)} = \frac1{s + \frac{1-s}{k^2}} \enspace .
% \end{equation}

% We used this scaling law as the graphical illustration for superlinear scaling in Fig.~\ref{fig:amdahl}. 
% For small values of $k$ we obtain $O(k^2)$ scaling, despite that uniform request distribution is the worst case for self-adjustments. This hints at a great future potential for networking workloads that typically exhibit highly skewed request distributions~\cite{832484}.



\subsection{The model}
\label{sec:model}

\paragraph{Architecture.}

Consider $k$ identical parallel machines $M_1, M_2, M_3, \ldots, M_k$, each having its own isolated memory and running an instance of a self-adjusting data structure $D$. A stream of requests $\sigma$ arrives to the load balancer that partitions the stream into $k$ streams $\sigma(M_1), \sigma(M_2), \ldots, \sigma(M_k)$ and dispatches them to the machines.

The load balancer is \emph{sticky}, i.e., it dispatches the requests to machines based solely on the request itself, and not on the state of the system.
Each request from $\sigma$ belongs to a universe $\mathcal{U}$. 
The load balancer is a function $f_k : \mathcal{U} \to \{1, 2, \ldots, k\}$ that partitions the universe $\mathcal{U}$ into $k$ subsets $\mathcal{U}_1, \mathcal{U}_2, \ldots, \mathcal{U}_k$ (often referred to as affinity domains).

\paragraph{Cost model.}

The time needed for the system to process a request $\sigma_t$ at time $t$ is the sum of computing the assignment of $\sigma_t$ (denoted $T(f(\sigma_t))$) and the time to schedule and process the request by $D$ at machine $M_f(\sigma_t)$ (denoted $T(D(\sigma_t)$).
The time $T(D(\sigma_t))$ depends on the internal state of the self-adjusting data structure $D$ at machine $f(\sigma_t)$.
Self-adjusting data structures change their internal state over time, hence for two identical requests coming at different times, the time to process them may differ.
The time required to process a request is often a function of the working set size of the data structure $D$. (TODO: cite)

The schedule finishes when all requests from $\sigma$ are processed. Some machines may be idle throughout execution, but the system is not allowed to reassign the requests to other machines.
We are interested in minimizing the completion time of the schedule.

% crucially, it's not only this:
% \[
%     \min \sum_t T(f(\sigma_t)) + T(D(\sigma_t)) \enspace .
% \]
% but more general schedule with gaps, but no reassignment


\subsection{The Scaling Law}

The definitions of $s$ and $T(1)$ are natural.
In similar fashion to Amdahl's law, we define $s$ as the serial fraction of the workload, where in our case only the load balancing is serial (total serial work is $\sum \sum_t T(f(\sigma_t))$).
Our benchmark $T(1)$ is a single self-adjusting data structure $D$ that runs on a single machine \footnote{A single self-adjusting data structure is the most natural baseline choice for the considered setting. But it differs from the baseline in Amdahl's law --- that would be a single machine simulating a non-trivial load balancer $f_k$ and $k$ independent machines to decrease the problem size. Another difference from the Amdahl's law assumptions is that the problem size is not fixed. The problem size at a machine, $T(f(\sigma_t))$, depends on the internal state of the self-adjusting data structure $D$ at time $t$ and machine $M_f(\sigma_t)$.}and processes the entire stream $\sigma$, with a trivial load balancer $f_1(\cdot) = M_1$.


Our law informally states the following: for any input $\sigma$, no self-adjusting distributed system adhering to our model can scale better than
\begin{equation*}\label{eq:mtf-perf}
  \frac{T(1)}{T(k)} \le \frac1{s + \frac{1-s}{k \cdot \ell}} \enspace ,
\end{equation*}
for some value $\ell$ that depends on how well the load balancer $f_k$ reduces the input $\sigma$.
We dedicate the Appendix~\ref{sec:apx} to the definition of $\ell$ and the proof of the theorem.
% We refer to the Appendix for formal definitions, cf. the proof of Theorem~\ref{thm:superlinear}.
The upper bound on scaling holds for all self-adjusting systems adhering to our architecture, which includes caching as a special case.




\paragraph{Consequences of the scaling law.}
Can we e.g. achieve quadratic scaling, i.e., $\ell = k$? 
Our theorem has implications that (informally) quadratic speedup is impossible to achieve for all $k$:
\begin{equation*}\label{eq:mtf-perf}
  \frac{T(1)}{T(k)} \neq \frac1{s + \frac{1-s}{k^2}} \enspace ,
\end{equation*}
The $r$ cannot grow indefinitely with $k$, and eventually the effects of load balancing ,,dry up". For each input sequence there exists a large enough value of $k$ so that adding machines no longer increases $\ell$ (details in Observation~\ref{obs:initial}). Hence, superlinear scaling is an initial phenomenon, and it cannot continue forever.





\section{Superlinear Scaling Theorem}
\label{sec:apx}

Our law informally stated the following impossibility result: no self-adjusting system can scale better than
\begin{equation*}
  S(k) = \frac{T(1)}{T(k)} = \frac1{s + \frac{1-s}{k \cdot \ell}} \enspace ,
\end{equation*}
for some value $\ell$ that depends on how well the load balancer $f_k$ reduces the input $\sigma$.

Before we state the theorem formally, we need to provide the formal definition of $\ell$.
The value of $\ell$ depends on the input $\sigma$, data structure $D$ and the load balancer $f_k$. Hence, $\ell$ indirectly relies on $k$ through $f_k$ (these are tied together in our architecture). Furthermore, for technical reasons, $\ell$ depends on the serial portion of the workload $s$.
To define $\ell$, we need to relate the input $\sigma$ with the load balancer $f_k$ and the data structure $D$.
We begin by defining the data structure $D$.



\paragraph*{Self-adjusting data structures and working sets.}
Our law captures various self-adjusting data structures, such as lists, caches and their generalizations.
In these data structures, the cost of accessing an item depends on the internal structure and changes over time depending on the history of requests.
Self-adjusting data structures have a property that the cost of accessing an item $x$ at time $t$ depends on the number of distinct items requested since the last access of $x$.
This is often referred to as the \emph{working set property}, and it can hold in an amortized sense.

With each data structure $D$, there exists an associated cost function $g_D$ 
\[
	\textsf{cost}(x, t, \sigma) = g_D(|W_t(x)|) \enspace ,
\]
where $W_t(x)$ is the number of distinct requests to items other than $x$ since the last request to $x$.
The working set property is usually an upper bound, but we state the law for the case where the cost is also \emph{lower-bounded} as a function of the working set size.
With this simplifying assumption, we still capture parallel extensions of LRU caches, Move-to-Front lists, and Move-Recursively-Forward packet classifiers, but not splay trees\footnote{todo: splay tree is only an upper bound, and holds in an amortized sense}. 
% The cost of rearrangement after access is usually proportional to access cost.

We illustrate $g_D$ for Move-to-Front and LRU. In Move-to-Front the cost of accessing an item is linear: $g_\textsf{MTF}(x,t,\sigma) = |W_t(x)| + 1$.  LRU is the algorithm Move-to-Front casted into the cost model of caching\footnote{See~\cite{SleatorT85} for the relationship between caching algorithms and self-adjusting list algorithms.}. Precisely, the cost model for $g_\textsf{LRU}$ is non-linear: for a cache of size $B$, the cost is 1 if the working set is larger than $B$, and 0 otherwise. Both cost functions $g_\textsf{MTF}$ and $g_\textsf{LRU}$ are monotonically increasing in the working set size. In fact, monotonicity of $g_D$ is a sufficient condition for our theorem to hold.

\paragraph*{Load balancer isolates working sets.}
Recall that the load balancer partitions the stream into $k$ streams $\sigma(M_1), \sigma(M_2), \ldots, \sigma(M_k)$ and dispatches them to the machines.
These streams may have reduced working set sizes at the machines in comparison to the working set sizes of the original stream $\sigma$.
Precisely, a load balancer $f_k : \mathcal{U} \to \{1, 2, \ldots, k\}$ partitions the universe $\mathcal{U}$ into $k$ subsets $\mathcal{U}_1, \mathcal{U}_2, \ldots, \mathcal{U}_k$, and the working set for machine $i$ at the time $x = \sigma_t$ is requested is $W_t(x, t, \sigma(M_i)) = W_t(x, t, \sigma) \cap \mathcal{U}_i$.

\paragraph*{Visualize the cost savings with a complete weighted graph.}
TODO

\paragraph*{Defining $\ell$: from an additive to a multiplicative law.}
In total, we observe cost savings from load balancing for the stream $\sigma$, data structure $D$ characterized by a function $g$, and a load balancer $f_k$. We denote them as
\[
	r = \sum_{t} g(|W_t(x)|) - \sum_{t} g(|W_t(x)| \cap \mathcal{U}_i) \enspace .
\]
Then, the total work decreases (a strict deviation from Amdahl's law) by $r$ for the stream $\sigma$. Let $r'$ be a normalized value of $r$ with respect to the total non-serial work of the system. Then, applying Amdahl's law to the reduced workload, we obtain the following theorem.


\begin{equation*}
  S(k) = \frac{T(1)}{T(k)} = \frac1{s + \frac{1-s - r'}{k}} \enspace ,
\end{equation*}
By setting $\ell = \frac{(1-s)\cdot k}{1-s-r'}$, we obtain the following theorem.

\begin{theorem}
	\label{thm:superlinear}
	Fix an input sequence $\sigma$.
	Assume that a self-adjusting data structure $D$ whose cost function depends on its working set size with a cost function $g_D$ that is monotonically increasing. 
	Then, for any number of machines $k$, the scaling of the system is upper-bounded by
	\begin{equation*}
	  S(k) = \frac{T(1)}{T(k)} = \frac1{s + \frac{1-s}{k \cdot \ell}} \enspace ,
	\end{equation*}
\end{theorem}
The proof of this theorem is a direct consequence of the Amdahl's law to the reduced workload. The total workload is reduced additively by $r$ (the exact value depends on $g_D$ and $f_k$), which gives the multiplicative law for our choice of $\ell$.


\paragraph*{Consequences.}
The above law has consequences for the scaling of self-adjusting systems. We express these consequences in terms of the additive $r$, but the consequences for $\ell$ are immediate.

The value of $r$ cannot grow indefinitely with~$k$.
We observe that after $k = |\mathcal{U}|$, all affinity domains are singletons, and increasing $k$ further does not reduce the cost any further, since there are no more working sets to isolate.
\begin{equation*}\label{eq:mtf-perf}
 \forall \sigma \exists k \frac{T(1)}{T(k)} \neq \frac1{s + \frac{1-s}{k^2}} \enspace ,
 TODO
\end{equation*}



\section{Initial Superlinear Scaling}

Finally, a positive result.
For some inputs, distributed self-adjusting list lookup can achieve \emph{initial} superlinear scaling.

\begin{observation}
	\label{obs:list-k2}
	In the distributed self-adjusting lookup, 
	for infinitely many inputs $\sigma$, there exists  and a constant $k_0$ such that all $k \le k_0$ there exists a load balancer $f_k$ that achieves have locality boosting savings $r' = \Omega(k^2)$.
\end{observation}

\begin{proof}
	Consider an input $\sigma$ that requests selected $k$ items of the universe in a round-robin fashion and repeats an arbitrarily large number of times $m$.
	The baseline $T(1)$ finishes the schedule in time $T(1) = 1$.

Consider $k$ machines. We construct a load balancer $f_k$.
First, we construct an auxiliary graph $G$ of $k$ vertices, where each vertex represents a machine. The graph is weighted, and each edge $(M_i, M_j)$ has the cost that is saved by isolating affinity domains between these two machines. For the special case of $\sigma$, this graph is a complete graph with equal weights.
Consider an ideal load balancer $f_k$ that isolates the affinity domains of all items from $\sigma$ by an arbitrary $k$-cut (components of size $k$)\cite{Frieze97, Mahajan95}. The total savings per each machine amount to $m\cdot(k-1)$.
Then, the distributed system pays $1$ for handling each request in the list lookup model, as opposed to $\Omega(k)$ of the baseline (all items are in the working set, and the function $g_D$ is linear).


TODO. explicitly show scaling by definition



% The distributed system with $k$ machines finishes its perfectly divisible parallel schedule (equal costs of all requests) in time $T(k) = s + W/k$, where $W$ is the total parallel work.


% Then, the total parallel work is 
% 	\[
% 		T(1) / T(k) \ge 1 / (s + (1-s-Omega(k^2))/k)  \enspace .
% 	\]


% The value $k_0$ must be chosen such that the load balancer has enough affinity domains. A trivial value of $k_0$ is $|\mathcal{U}|$, but for inputs with high locality $k_0$ can be smaller. 
\end{proof}

\section{A Stronger Scaling Law for Lists and Packet Classification}

\paragraph*{Strong guarantees for list algorithms.}
Consider a generalized architecture, where we can change the data structure $D$ on each machine $M_i$ to any other data structure $D_i$, even to the offline optimum for the given sequence. 

cite infocom and sleator papers


Then, the analysis of Albers shows that the system cannot improve more than a constant factor above the Move-to-Front algorithm (there exist competitive bounds).



%%% Local Variables:
%%% mode: latex
%%% TeX-master: "distributed_mrf"
%%% End:
